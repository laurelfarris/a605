\documentclass[12pt]{article}
%\usepackage[left=1in, top=1in, right=1in, bottom=1in]{geometry}
\usepackage[textwidth=5.5in]{geometry}
%\renewcommand\familydefault{\sfdefault}
\setlength{\marginparwidth}{0.5in}
\setlength{\parindent}{0em}
\usepackage{mathpazo}
\usepackage{graphicx}
\usepackage{framed}
\usepackage{lipsum}
\usepackage{enumitem}
\usepackage{amsmath}
\usepackage{amssymb}
\usepackage{xcolor}

\usepackage{fancyhdr}
\pagestyle{fancy}
\fancyhf{}  % Clear all headers and footers (including default page number).
\renewcommand{\headrulewidth}{0pt}
\rfoot{\thepage}

\definecolor{mygray}{rgb}{0.43, 0.5, 0.5}
\usepackage{mathtools}  % \Aboxed{}, among other things...
\usepackage{ragged2e}
\newlength\ubwidth
\newcommand\parunderbrace[2]{%
    \settowidth\ubwidth{$\quad#1\quad$}
    \begingroup\color{mygray}\underbrace{\color{black}#1}_{%
    \parbox{\ubwidth}{\scriptsize\centering#2}}\endgroup
}

\usepackage[symbol]{footmisc}
\usepackage{perpage}
\MakePerPage{footnote}
%\renewcommand\footnoterule{\rule{\textwidth}{0.4pt}}
\renewcommand{\footnoterule}{
  \kern -3pt
  \hrule width \textwidth height 0.4pt
  \kern 2pt
}

\usepackage{marginnote}
%\renewcommand*{\raggedrightmarginnote}{\centering}
\renewcommand*{\raggedleftmarginnote}{\centering}
\newcommand{\mar}[1]{\hspace{0pt}\marginpar{-\textcolor{black}{#1}-}}

\definecolor{bred}{rgb}{0.8, 0.0, 0.0}
\definecolor{mygreen}{rgb}{0.0, 0.26, 0.15}
\newcommand{\mynotes}[1]{{\fontfamily{cmss}\selectfont \textit{#1}}}
%\newcommand{\mynotes}[1]{\textcolor{mygreen}{#1}}

\newcommand{\mydate}[1]{
    \rule{\textwidth}{0.4pt}

    {\vspace{-1ex}\small\hfill\textit{#1}}}


\usepackage{titlesec}
%\titleformat{<command>}
%   [<shape>]{<format>}{<label>}{<sep>}{<before-code>}[<after-code>]
\titleformat{\section}%
    [hang]
  {\filcenter\fontsize{16}{18}\selectfont\bfseries} %\filcenter\bfseries\LARGE
  {\hspace{-0.25in}\arabic{section}.\;} %   {\thesection} %{<label>}
  {1em}
  {}
\titleformat{\subsection}%
  {\filcenter\fontsize{14}{16}\selectfont\bfseries} %\filcenter\bfseries\LARGE
  {\arabic{section}.\arabic{subsection}\;}
  {1em}     % sep
  {}        % before code
\titleformat{\subsubsection}%
  {\filcenter\fontsize{13}{15}\selectfont\bfseries\itshape} %\filcenter\bfseries\LARGE
  {\arabic{section}.\arabic{subsection}.\arabic{subsubsection}\;}% label% {\thesection} %{<label>}
  {1em}     % sep
  {}        % before code
\titleformat{\paragraph}%
  {\fontsize{12}{13}\selectfont\bfseries}
  {$\Rightarrow$}{}{}

\titlespacing*{\section}{0pt}{0pt}{12pt}
\titlespacing*{\subsection}{0pt}{10pt}{2pt}
\titlespacing*{\subsubsection}{0pt}{10pt}{2pt}
\titlespacing*{\paragraph}{0pt}{1ex}{-2ex}

% Start new page for every section
%\let\stdsection\section
%\renewcommand\section{\newpage\stdsection}
\let\oldsection\section
\renewcommand\section{\clearpage\oldsection}

% Section references
%\renewcommand{\thesection}{}
%\renewcommand{\thesubsection}{\arabic{subsection}}
%\renewcommand{\thesubsubsection}{\arabic{subsubsection}}

%\setcounter{secnumdepth}{1}

\setlist[itemize]{noitemsep, topsep=-1ex}
\setlist[enumerate]{noitemsep, topsep=-1ex}
\renewcommand{\labelitemi}{$\vcenter{\hbox{\scriptsize$\bullet$}}$}
\renewcommand{\labelitemii}{$\vcenter{\hbox{\scriptsize$\circ$}}$}
%\renewcommand{\labelitemi}{{\tiny$\bullet$}}
\definecolor{cadet}{rgb}{0.33, 0.41, 0.47}
\renewcommand{\descriptionlabel}[1]{%
    \bfseries\textcolor{cadet}{#1}}

\usepackage{listings}
\lstset{% general command to set parameter(s)
    basicstyle=\small, % print whole listing small
    keywordstyle=\color{black}\bfseries\underbar,% underlined bold black keywords % nothing happens
    identifierstyle=,
    commentstyle=\color{white},
    stringstyle=\ttfamily,
    showstringspaces=false % no special string spaces
    }

%\usepackage{setspace} % spacing between toc items
%\usepackage[toc]{multitoc}
%\renewcommand*{\multicolumntoc}{2}
%\setlength{\columnseprule}{0.5pt}

\usepackage{hyperref}
\definecolor{darkpowderblue}{rgb}{0.0, 0.2, 0.6}
\hypersetup{colorlinks=true, urlcolor=darkpowderblue, linkcolor=black}
\urlstyle{same}

\begin{document}
\setlength{\parskip}{0ex}
\tableofcontents\newpage
\setlength{\parskip}{2ex}

\reversemarginpar
\section{Introduction}
\subsection{Overview of ISM components and processes}
\mar{6}What can happen to an atom or molecule or dust
grain sitting in the ISM?
\begin{itemize}
    \item Can it absorb a photon? Energy levels $\leftrightarrow$ $h\nu$
        (Need cross-sections for dust grains)
    \item Can it collide with other particles? $\rightarrow$ collisional
        rates [s$^{-1}$ cm$^{-3}$]
    \item Is there a magnetic field? Is the particle charged?
    \item Are there cosmic rays? These can penetrate dense gas.
\end{itemize}
Three possible sources of ionization (and excitation):
\begin{enumerate}
    \item photons
    \item collisions
    \item cosmic rays
\end{enumerate}
Typical collisional energies: kinetic energy.
Translate $mv^{2} \rightarrow kT \rightarrow h\nu \rightarrow eV$.

\subsection{Dust grains}
\mar{8}Dust grains also occur in the neutral medium, and
probably also in the (warm) ionized medium. Dust grains play an important
role in various processes:
\begin{itemize}
    \item extinction of starlight
    \item emission of absorbed energy in FIR
    \item formation of molecules often occurs on grain surfaces
    \item absorption of ionizing UV radiation and Ly$\alpha$ photons
        (reducing amount of ionizing radiation)
    \item heating of HI gas by \emph{photoelectric} emission
\end{itemize}
Composition: carbon and silicates. Typical sizes: 0.01 - 0.1 $\mu$m
(How do we know? $\rightarrow$ shape of extinction curve). Grains as small
as $\sim$ 60 atoms across discovered; evidence from emission lines in NIR
and excess emission at 5 - 40 $\mu$m over what is expected from dust in the ISM\@.
The larger dust grains have temperatures between 10 and 40 K, while the small
ones can be heated to higher temps due to the absorption of even a single
photon (smaller heat capacity, as volume $\propto r^{3}$). A promising
candidate for small dust grains: polycyclic aromatic hydrocarbons
($\sim$ car soot!)

In hot environments, dust grains may be destroyed by \textit{sputtering}, where
collisiosn of grains with other atoms, electrons, or molecules knock molecules
off the grains. \mar{9}At low temperatures, molecules stick to dust grains,
causing depletion of heavy elements along certain lines of sight (most dust in
the plane). Dust contributes about 1\% of the mass of the ISM in the solar
neighborhood, mostly in the form of large grains.

\subsection{Ionized gas}
\begin{enumerate}[label=\alph*)]
    \item Photoionization: especially effective near hot stars. Shock
        ionization
    \item Cosmic rays: can occur throughout most of the ISM, so can also
        produce a small amount of ionization in denser gas (though
        recombination happens quickly, so not much of gas is in an
        ionized state at any given time).
    \item \mar{11}Collisions: Hot ($\sim 10^{7}$ K) expanding bubble
        sweeps up shell of warm ($\sim 10^{4}$ K) ionized gas, which shows
        a different optical spectrum than HII regions. The hot gas shows
        up by:
        \begin{itemize}
            \item Free-free and line x-rays
            \item absorption lines of highly ionized species toward
                bright UV sources
        \end{itemize}
        Kirchoff's law: apparently can still see absorption lines when looking
        through a gas that is hotter than the source behind it!
\end{enumerate}

\subsection{Magnetic fields and cosmic rays}
\mar{13}In the solar neighborhood, $\mathbf{B} \sim 2-5\times10^{6}$ Gauss.
This follows from measurements of \textit{Faraday rotation}, giving
$<n_{e}B_{||}>$ toward pulsars and radio sources. The random component of
the $B$ field is probably as large as the uniform component. In clouds,
the $B$ field can be much higher, $\sim 70 \mu$G (from Zeeman effect splitting
measurements).

The magnetic field is important for several reasons:
\begin{enumerate}
    \item It aligns elongated grains, giving rise to polarization
        of starlight
    \item Causes relativisitic electrons to emit synchrotron radiation,
        and most likely plays a role in accelerating electrons to
        relativistic velocities (``magnetic bottle'', Fermi acceleration).
    \item Provides pressure support against gravitational collapse of matter
        since it is frozen into the matter due to ionization heavy elements.
        It also seems to play animportant role in solving the angular momentum
        problem in star formation.\footnote{Shu et al.}
\end{enumerate}

Total energy density of cosmic rays in solar neighborhood:
$U_{R} \sim 1.3\times10^{-12}$ erg cm$^{-3}$.
\mar{14}Why are cosmic rays important?
\begin{itemize}
    \item They produce $\gamma$-rays through collisions with atoms and
        molecules. The observed $\gamma$-ray intensity from the ISM forms
        an excellent independent measure of the total amount of matter
        between stars. For example, calibrating the conversion factor of
        CO line intensity to H$_{2}$ mass.
    \item Provide pressure against gravitational collapse
\end{itemize}
Five pressures that play an important role in supporting the ISM against
gravitational collapse:
\begin{enumerate}
    \item thermal $P=nkT$
    \item magnetic $P=\cfrac{B^{2}}{8\pi}$
    \item turbulent (bulk motion)*
    \item cosmic ray
    \item radiation
\end{enumerate}
* The cloud to cloud velocity dispersion due to turbulence on various
scales increases line widths [over?] thermal widths. From the table
on page -12-, you will note rough thermal pressure euqilibrium between
the components. This is not a coincidence; in fact, some of this
information was inferred by assuming pressure equilibrium. The
argument is that if there were no equilibrium, the resulting
perturbations would be wiped out on sound-crossing time scales, which
are short compared to\mar{15} the time scales we wouuld
consider the ISM to evolve over.

However, the actual evidence for equilibrium in the thermal pressure
is scarse, and there are claims that it is not true in the very local
ISM\footnote{\href{http://www.nature.com/nature/journal/v375/n6528/abs/375212a0.html}
{Bowyer et al. \textit{Nature} 1905}}.

There seems to be a ``cosmic conspiracy'': the estimates for the
thermal, magnetic, and cosmic ray pressure for the solar neighborhood
give roughly equal numbers for all three. Thus it may be inappropriate
to only consider the thermal pressure (the only one that can be
measured with much certainty). Interestingly, the magnetic pressure
number is also very similar to the energy density of the CMB\@.
\footnote{Draine discusses possible reasons in section 1.3}
\footnote{Supplemental info in Draine chapter 1.}

Next\mar{16} section is sort of a shortened condensation of Draine's chapters
2 and 3. We may come back to specific topics discussed there in more
detail.

\newpage
\section{Statistical physics in the ISM}
\subsection{The big four}
The four major laws of statistical physics are:
\begin{enumerate}
    \item \textbf{Maxwellian} velocity distribution
    \item \textbf{Boltzmann distribution} of energy levels in atoms and molecules
    \item \textbf{Saha equation} for ionization equilibrium
    \item \textbf{Planck function} for radiation
\end{enumerate}
\mynotes{These apply to stars, but not always to the ISM.}

\subsubsection{Maxwellian}
\mynotes{$T$ is defined by motion of particles}.
($\vec{\omega}$ = velocity = $\vec{v}$ in Draine).

$f(\vec{\omega})\mathrm{d}\vec{\omega}$ =
fractional number of particles whose velocity, $\vec{\omega}$, lies within the
three-dimensional volume element
$\mathrm{d}\vec{\omega} =
\mathrm{d}\omega_{x}\mathrm{d}\omega_{y}\mathrm{d}\omega_{z}$,
centered at $\vec{\omega}$.

In thermodynamic equilibrium, $f(\vec{\omega})$ is isotropic, so
$\vec{\omega} \rightarrow \omega$.
\[
    f(\omega) =
    \frac{\ell^{3/2}}{\pi^{3/2}} e^{-\ell^{2}\omega^{2}}; \quad
    \ell^{2} = \frac{m}{2kT} = \frac{3}{2 \langle \omega^{2} \rangle }
    \]
\[
    f(\omega) = \left(\frac{m}{2\pi kT}\right)^{3/2}
    \exp\left(-\frac{m\omega^{2}}{2kT}\right)
    \]
For two groups of particles with different masses, we replace $\omega$
by $u$, the relative velocity between the two groups, and $m$ by the
reduced mass $m_{r} = \cfrac{m_{1}m_{2}}{m_{1}+m_{2}}$.
For\mar{17} H atoms colliding with particles of mass $Am_{H}$,
Spitzer derives:
\[
    \langle u \rangle
    = \left[ \frac{8kT}{\pi m_{r}} \right] ^{1/2}
    = 1.46\times10^{4}\sqrt{T} \left( 1+\frac{1}{A} \right) ^{1/2}\;
    [\mathrm{cm\; s}^{-1}]
    \]
\begin{framed}
    \underline{Exercise}: Verify the above calculation, especially the
    numerical constant.
\end{framed}
Note that there is a difference between the \emph{speed} and
\emph{velocity} distribution. Verify that $\langle \omega \rangle$ = 0.
But evidently, the mean \emph{speed} is not 0. The speed distribution is
given by:
\[
    f' \left( \omega' \right) \mathrm{d}\omega'
    = \left( \frac{m}{2{\pi}kT} \right) ^{3/2}
    \exp \left( -m\omega'^{2}/2kT \right)
    \parunderbrace{
        4\pi\omega'^{2}\mathrm{d}\omega'}{
        volume in phase space}
    \]
where $f' \left( \omega' \right) \mathrm{d}\omega'$ = fractional number of
particles with speeds between $\omega'$ and $\omega' + \mathrm{d}\omega'$
and $\omega' = | \omega |$.

The Maxwell velocity distribution is characterized by several speeds:
\begin{itemize}[itemsep=1ex]
    \item Most probable speed: $\omega_{o} = \sqrt{\cfrac{2kT}{m}}$
    \item RMS speed: $<\omega^{2}>^{1/2} = \sqrt{\cfrac{3kT}{m}}$
    \item RMS velocity in one direction:
        $<\omega_{x}^{2}>^{1/2} = \sqrt{\cfrac{kT}{m}}$
\end{itemize}

\subsubsection{Boltzmann distribution}
\mar{18}The Boltzmann distribution gives the population of energy levels in
an atom or molecule:
\[
    \frac{n_{u}}{n_{l}} =
    \frac{g_{u}}{g_{l}}\exp\left[- \left( E_{u}-E_{l} \right) /kT\right] =
    \frac{g_{u}}{g_{l}}\exp\left[-h\nu_{o}/kT\right]
    \]
where
\begin{itemize}
    \item $n_{u,l}$ are the number densities
    \item $g_{u,l}$ are the statistical weights
    \item $E_{u,l}$ are the energies of the levels
\end{itemize}
\mynotes{(Partition function: summing over all energy levels\ldots chemical
term).}

\subsubsection{Saha equation}
The Saha equation describes ionization equilibrium:
\[
    \frac{n_{i+1}}{n_{i}} = \frac{g_{i+1}g_{e}}{g_{i}}
    \left( \frac{2\pi m_{e}kT}{h^{2}} \right) ^{3/2}
    \exp \left( -I/kt \right)
    \]
where $I$ is the ionization potential for an ion in the ground state
and initial ionization state $i$ (aka, the energy required to ionize
from $i$ to $i+1$). $g_{e} = 2$ (two spin conditions).
\mynotes{Electrons affect whether and how easily atoms can be ionized.}

\subsubsection{Planck function} specifies the radiation field:
\footnote{
    $e^{x} \approx 1 + x$ for $x \ll 1$\\
    $e^{h\nu/kT}-1 \approx 1$ for $h\nu \ll kT$\\
    $e^{h\nu/kT}-1 \approx e^{h\nu/kT}$ for $h\nu \gg kT$}
\begin{align*}
    B(\nu) &= \;\frac{2h\nu^{3}}{c^{2}}\frac{1}{\exp[h\nu/kT]-1}\\
    &= \;\sim \frac{2\nu^{2}}{c^{2}}kT\quad \mathrm{for}\;h\nu \ll kT\;
    (\mathrm{Rayleigh-Jeans})\\
    &= \;\sim \frac{2h\nu^{3}}{c^{2}}\exp[-h\nu/kT]\quad\mathrm{for}\;h\nu \gg kT
    (\mathrm{Wien})
\end{align*}

\subsection{Validity of the four laws}
\mar{19}The four laws discussed above hold under thermodynamic equilibrium
(TE). However, TE is not often the case for the ISM, for two reasons:

\begin{enumerate}
    \item TE requires \textbf{detailed balancing},
        i.e.\ each process is as likely to occur as its inverse. For
        example, consider the 3727 \AA{} emission from O$^{+}$. This is a
        forbidden transition (actually a doublet). The excitation of the
        electron level occurs through collisions with electrons, in most
        conditions in the ISM\@. If detailed balancing were to hold,
        de-excitation should also occur by collisions. However, as we will
        see, under the low density conditions found in the ISM, collisions
        are rare, and de-excitation is more likely to proceed through
        emission of a photon, in spite of the fact that we are dealing with
        a forbidden transition. Thus [OII] emission can be quite strong,
        and by converting collisional (kinetic) energy into radiation, we
        actually have created a cooling mechanism for the gas.
    \item The radition field is strongly diluted. A diluted radiation field
        is one in which the energy density does not match the color
        temperature. This concept is quite familiar; for example, the sun's
        photosphere is $\sim$ 6000 K, and at the surface the flux leaving
        the sun is approximately that of a blackbody of this temperature.
        However, the Earth is not 6000 K because by the time the radiation
        reaches us, it is diluted. \mar{20}For the solar neighborhood, the
        total energy density of the radiation field due to all stars in
        that volume is about 1 eV cm$^{-3}$ (close to cosmic ray density,
        as mentioned before). When interpreted as an average temperature
        using the Stefan-Boltzmann law (energy density of a blackbody, u =
        aT$^{4}$), this energy density implies an equivalent temperature of
        $\sim$ 3 K. Yet the color temperature implied by the shape of the
        spectrum of this Interstellar Radiation Field (ISRF) is that of A
        and B stars (T $\sim$ 10$^{4}$ K). So there is a \textbf{dilution
        factor} $W$ given by:
        \[
            W \approx \left( \frac{3}{10^{4}} \right) ^{4}
              \approx 0.25\times10^{-14}
            \]
        We conclude that using the Planck law to
        describe intensities is not correct.
\end{enumerate}
What about the other laws?

\paragraph{1. Maxwell velocity distribution} Good news! It is generally
valid. Detailed balancing is possible for the elastic collisions that are
generally occurring.\footnote{
    See Draine section 2.3.3 for an analysis of ``deflection times''; they
    tend to be short. (In dynamics, we speak of a ``relaxation time''.)}
Because the maxwellian distribution is a good
description of the motions of the particles, we can define a \textit{kinetic
temperature} which describes the physical condition of the gas. Often,
for a plasma, the kinetic temperature is equal to the electron temperature:
$T_{ions} = T_{e}$. $T_{ions} \neq T_{e}$ may occur behind shocks.

\paragraph{2. Boltzmann distribution}
\mar{21}Rarely correct.
If excitation and de-excitation occurs by photons, we may still not have a
Boltzmann distribution because the photon distribution is not given by the
Planck function. Often we do not even have detailed balancing.
However, as we will see, sometimes the distribution of excited levels
is not too different from Boltzmann distribution. This happens when collisions
dominate excitation and de-excitation, while radiation is relatively unimportant.

To describe situations close to TE, Spitzer introduced the so-called
b-factors (Draine calls them ``departure coefficients'').
\[
    b_{j} \equiv
    \frac{n_{j}(\mathrm{true\; distribution}}{n_{j}(\mathrm{LTE\; distribution})}
    \]
Example: in an HII region, the hightest excited levels of HI have $b_{j} \sim 1$.
Some radiation does escape (producing radio recombination lines) but collisions
dominate the level populations. Since motions of particles \emph{are} described
by a Maxwellian velocity distribution, whenever collisions dominate the level
population they will closely follow a Boltzmann law.

\mar{22}In general, a Maxwellian velocity distribution tends to set up a
Boltzmann population for energy levels in the atoms/particles \emph{if}
transitions resulting from emission and absorption of photons are
relatively unimportant, and collisional (de-)excitation is dominant. In the
case of the highly excited levels in H mentioned before, collisions with
electrons are dominant.

\paragraph{3. Saha equation} is generally not valid; there is no detailed
balancing. Even though the ionization and recombination processes are
each other's inverse
\newline
($h\nu + A \rightarrow A^{+} + e^{-} \rightarrow h\nu + A$),
\newline
the ionization process is determined by the \emph{photon} field in most
cases, while the recombination process is determined by collisions between
A$^{+}$ and e$^{-}$. The collision rate depends on \{$n_{e}, n_{A^{+}}$\}
and $T_{e}$ (the electron temperature), but the ionization is dependent on
$T_{radiation} (\neq T_{e})$.

So in general, assume \textbf{statistical equilibrium}, where there is
a balance between transitions one way and the other way, no matter
what process caused each transition.

In level $i$, we have $n_{i}$ atoms cm$^{-3}$, and
$R_{ij}$ is the rate coefficient (number of transitions from level $i$
to level $j$ for all possible processes) such that

$n_{i}R_{ij}$ [s$^{-1}$] = \# transitions from level $i$ to level $j$\\
$R_{ji}n_{j}$ [s$^{-1}$] = \# transitions from level $j$ to level $i$\\

\[
    \frac{\mathrm{d}n_{i}}{\mathrm{d}t} =
    \sum_{j}\left(-R_{ij}n_{i} + R_{ji}n_{j}\right);\quad i=1,2,\ldots
    \]
\mar{23}In statistical equilibrium,
$\cfrac{\mathrm{d}n_{i}}{\mathrm{d}t} = 0$. The rate factor $R_{ij}$
includes \emph{all} possible processes that would take the atom or
molecule from level $i$ to $j$.
For example, if you consider excitation of an electron in an atom,
$R_{ij}$ could include absorption of photons, collisional excitation, etc.

In the worst case, you will have to include many processes to calculate the
$n_{i}$ values. This requires knowledge of a lot of physical input
parameters, e.g.\ cross sections for particular processes, collisional rate
coefficients \footnote{Drane section 2.1}, etc.

In other cases, where only one or two processes matter, the situation
can be very simple. We will encounter cases of each.

Before going more into Ch 2 and 3 in Draine, we will first discuss
some basic radiative transfer.\footnote{RL Ch 1, Draine Ch 6, 7}

\newpage
\section{Radiative Transfer}
\mar{24}(See Draine chapter 7)
\begin{description}[align=right, labelwidth=10em, labelsep=3em, leftmargin=13em,
        itemsep=1ex, topsep=1ex]
    \item [Planck function (an intensity)]
        $ B_{\nu}(T) \equiv
        \cfrac{2h{\nu}^{3}}{c^{2}} \cfrac{1}{e^{h{\nu}/kT}-1} \;
        \cfrac{dE}{dt d\nu d\Omega d\sigma} $
    \item [Photon occupation number]
        $ n_{\gamma} \equiv \cfrac{c^{2}}{2h{\nu}^{3}} I_{\nu}
          = \cfrac{1}{e^{h{\nu}/kt} - 1} \quad \mathrm{if}\; I_{\nu} = B_{\nu} $
    \item [Specific intensity]
        $ I_{\nu} \equiv \lim_{d\sigma, d\Omega, d\nu, dt \to 0} $
    \item [Flux at surface of a sphere]
        $ F_{\nu} = \pi{B_{\nu}} $
        (for blackbody)\newline
        $ F_{\nu} = \pi{I_{\nu}} $
        (for isotropic emitting non-blackbody)
    \item [Flux at a distance $r$]
        $ F_{\nu}(r) = \pi{I_{\nu}}(\cfrac{R}{r})^{2}
        = \cfrac{L_{\nu}}{4\pi{r^{2}}}$ \newline
        where $R$ = radius of body and \newline
        $L_{\nu}$ = luminosity of body
        [erg s$^{-1}$ Hz$^{-1}$]
\end{description}


\mar{25}The above are \textit{monochromatic}: integration over frequency
yields the total flux, etc.
\[
    I = \int_{r=0}^{\infty} I_{\nu} \mathrm{d}\nu \quad
    F = \int_{\nu=0}^{\infty} F_{\nu} \mathrm{d}\nu \;\mathrm{etc\ldots}
    \]

Energy density:
$u_{\nu} = \frac{1}{c}\int{I_{\nu}\mathrm{d}\Omega}$ [erg cm$^{-3}$
Hz$^{-1}$]

Radiation pressure:
$P_{\nu} = \frac{1}{c}\int{I_{\nu}\cos^{2}\theta\mathrm{d}\Omega}$

\newpage
\subsection{Emission and absorption coefficients}
\subsubsection{Spontaneous Emission}
\begin{itemize}[label={}, itemsep=0ex]
    \item $j_{\nu}$ = volume emission coefficient
        [erg cm$^{-3}$ sec$^{-1}$ $\Omega^{-1}$ Hz$^{-1}$]
    \item $j'_{\nu} = \cfrac{j_{\nu}}{\rho}$ = mass emission coefficient
        [erg g$^{-1}$ sec$^{-1}$ $\Omega^{-1}$ Hz$^{-1}$]
    \item $j_{\nu} = \cfrac{\epsilon_{\nu}}{4\pi}$ for an isotropic
        emitter;
        $\epsilon_{\nu}$ = emissivity [erg cm$^{-3}$ s$^{-1}$ Hz$^{-1}$]
\end{itemize}

\subsubsection{Absorption}
Absorption is a bit more complicated than emission. It removes a fraction
of the incoming radiation. Absorption also includes \textit{stimulated}
emission, aka ``negative absorption''. The entire process refers to the sum
of ``true absorption'' + stimulated emission.
\begin{itemize}[label={}, itemsep=0ex]
    \item \mar{26}$\kappa_{\nu}$ =  volume absorption coefficient [cm$^{-1}$]
    \item $\kappa'_{\nu} = \cfrac{\kappa_{\nu}}{\rho}$
        = mass emission coefficient [g$^{-1}$ cm$^{2}$]
\end{itemize}
Loss of intensity in a beam of light as it travels distance $\mathrm{d}s$:
\[
    \mathrm{d}I_{\nu} = -\kappa_{\nu}I_{\nu}\mathrm{d}s
    \]
Microscopically:
\[
    \kappa_{\nu} = n\sigma_{\nu}
    \]
where $n$ is the particle density and $\sigma_{\nu}$ is the cross section
for the absorption process per particle.

\textbf{optical depth}: $ \tau_{\nu} \equiv \int{\kappa_{\nu}ds} $
\begin{itemize}
    \item $\tau \gtrsim 1$ optically \emph{thick} emission
    \item $\tau < 1$ optically \emph{thin} emission
\end{itemize}

\textbf{mean free path}: $ \ell_{\nu} = \cfrac{\tau_{\nu}}{\kappa_{\nu}} $

so for condition $\tau=1 \longrightarrow \ell_{\nu} = \cfrac{1}{\kappa_{\nu}} =
\cfrac{1}{n\sigma_{\nu}}$

\newpage
\subsection{Radiative transfer equation}
\begin{align*}
    \frac{\mathrm{d}I_{\nu}}{\mathrm{d}s} &= -\kappa_{\nu}I_{\nu} + j_{\nu}\\
    \mathrm{d}\tau_{\nu} &= \kappa_{\nu}\mathrm{d}s\\
    \frac{\mathrm{d}I_{\nu}}{\mathrm{d}\tau_{\nu}} &=
    -I_{\nu} + \frac{j_{\nu}}{\kappa_{\nu}} = -I_{\nu} + S_{\nu}
\end{align*}
where $S_{\nu} \equiv$ \underline{source function}

\paragraph{Formal solution:}
\[
    I_{\nu}(\tau_{\nu}) = I_{\nu}(o)\mathrm{e}^{-\tau_{\nu}} +
        \int_{0}^{\tau_{\nu}}\!{\mathrm{e}^{-(\tau_{\nu}-\tau'_{\nu})}
        S(\tau'_{\nu})\mathrm{d}\tau'_{\nu}}
        \]
\mar{27}$\rightarrow$ attenuated incoming beam + contribution from gas itself.

\paragraph{Special cases:}
\begin{enumerate}
    \item Source function is constant throughout source:
        \[
            I_{\nu}(\tau_{\nu}) =
            I_{\nu}(o)\mathrm{e}^{\tau_{\nu}}
            + S_{\nu}(1-\mathrm{e}^{-\tau_{\nu}})
            \]
        \begin{itemize}[itemsep=1ex]
            \item optically thick emission:
                $I_{\nu}=S_{\nu}$
            \item optically thin emission:
                $I_{\nu}=I_{\nu}(o)(1-\tau_{\nu}) + \tau_{\nu}S_{\nu}$
        \end{itemize}
    \item Thermal radiation: $S_{\nu} = B_{\nu}(T) $ (The Planck function)
        \begin{itemize}[itemsep=1ex]
            \item optically thick emission:
                $I_{\nu}=B_{\nu}$
            \item optically thin emission:
                $I_{\nu}=\tau_{\nu}B_{\nu}$
        \end{itemize}
\end{enumerate}

\paragraph{Brightness temperature}
For \emph{radio} emission:
$I_{\nu}$ is replaced by the \textit{brightness temperature},
$T_{b}$, defined as:
\[
    I_{\nu,\mathrm{obs}} \equiv B_{\nu}(T_b)
    \]
In the Rayleigh-Jeans limit for $B_{\nu}$ we get:
\[
    I_{\nu,\mathrm{obs}} = \frac{2\nu^{2}kT_{b}}{c^{2}}
    \quad\longrightarrow\quad
    I_{b} = \frac{I_{\nu}c^{2}}{2\nu^{2}k}
    \]
Solution of the transfer equation in terms of $T_{b}$:
\[
    T_{b,\mathrm{obs}} =
    T_{b,o}\mathrm{e}^{-\tau_{\nu}}
    + T(1-\mathrm{e}^{-\tau_{\nu}})
    \]
\begin{itemize}[label={}, noitemsep]
    \item \mar{28}$T$ = thermal source temperature (\emph{physical}
        temperature of the layer)
    \item $T_{b}$ = brightness temperature of the incident radiation
\end{itemize}
\textcolor{bred}{$T_{b}$ is never greater than $T$!}

\subsection{Einstein coefficients}
At the macroscopic level, Kirchoff's law gives $ j_{\nu} = \kappa_{\nu}
B_{\nu}(T) $. Einstein coefficients\footnote{Draine, section 6.1} describe
reasons for this on a microscopic level. They give the transition
\emph{probabilities} (per unit time). Three possible processes:
\begin{enumerate}[noitemsep]
    \item \underline{Spontaneous emission}: Einstein $A$ coefficient.
        $A_{21} \equiv$ transition probability for spontaneous emission
        per unit time per ``system''.
    \item Absorption
    \item \mar{29}Stimulated emission
\end{enumerate}

\subsubsection{Relations between the Einstein coefficients}
Valid under all conditions since they only refer to \emph{atomic} principles
(no collisions, just radiation).
TE: rate of transitions out of state 1 = rate of transitions into state 1
(per unit volume).

In\mar{30} TE, apply the Boltzmann law.

\subsubsection{Relations between Einstein coefficients and $\kappa_{\nu}$ and $j_{\nu}$}
\paragraph{emission coefficient}
\paragraph{absorption coefficient}

The\mar{31} second term here for $\kappa_{\nu}$ corresponds to
\emph{stimulated emission}.

\subsection{Line profile function, $\phi(\nu)$}
See RL chapter 10.6 and Draine 6.4. In particluar, comments in Draine about
the actual line width in some practical cases are useful!

\subsubsection{Natural line width}
\mynotes{Atom not moving, but still quantum effects.}
Determined by lifetime of an excited state. Described by Lorentz profile,
whose key feature is a narrow core and broad wings (not a Gaussian).
\[
    \phi(\nu)
    = \frac{4\gamma_{u\ell}}{16\pi^{2} (\nu - \nu_{u\ell})^{2} +
    \gamma_{u\ell}^{2} }
    \]
\underline{Key point}: A small Einstein coefficient $A$ results
in a \emph{narrow} line.

The natural line width of most transitions is quite small, and broadening
due to other effects is more important.
\subsubsection{Doppler broadening}
\begin{itemize}
    \item Thermal velocities
    \item Bulk motion (turbulence)
\end{itemize}

and\mar{32} the profile function is

\subsubsection{Collisional broadening}
$\sim$ Pressure broadening, which is not generally important in the ISM because
the density is so low\ldots mostly occurs in stellar atmospheres.
This still produces a Lorenzian profile, but with:
\[
    \phi(\nu) = \frac{4\Gamma^{2}}{16\pi^{2}(v-v_{o})^{2} + \Gamma^{2}}
    \]
Can be written in terms of a \emph{Voigt function}:

\mar{33}So the core of the profile is Gaussian due to Doppler broadening,
while the wings are much stronger than expected in a Gaussian profile,
due to the intrinsic line width.

\newpage
\section{Neutral hydrogen (HI gas) in the ISM}
\footnote{Draine Ch 8, 29; Ch 17.1, 17.3}
Wherever HI dominates the ISM, all atoms are found in the ground state
($^{2}$S)(n=1). The next excited level ($^{2}$P) is about 10 eV above the
ground-level. This excitation is very rare and and quickly falls back to the
ground level, so there is no significant population of this level. For example,
consider potential excitation mechanism: collision of H-atom with cosmic ray
particle (lots of energy, probability is once per 10$^{17}$ seconds), ionizes
the H-atom. Recombination results in some atoms winding up in $^{2}$P state.
But the Einstein $A$ coefficient for spontaneous emission from $^{2}P$
$\rightarrow$ $^{2}S$ is 10$^{8}$ s$^{-1}$. Hence, this excitation process
results in relative population of $^{2}P$ level of $10^{-8}/10^{17} =
10^{-25}$.

ISM is too cool for collisions to happen often and cosmic rays are rare.

\subsection{Possible tracers of HI gas}
\begin{enumerate}[noitemsep]
    \item 21 cm HI transition (=hyperfine transition) in emission or absorption.
    \item Lyman absorption lines against hot background stars.
\end{enumerate}
Why only these two? Because only the $^{2}$S level is populated. HI is hard to
find in the ground state; fine structure $\rightarrow$ different angular
momentum.\footnote{
    See Draine Ch 4 (\& 5) on notation of energy levels and
    atomic structure.}

\newpage
\subsubsection{Excitation and radiative transport for the 21-cm line}
\mar{34}Spin of proton and electron:
\begin{itemize}
    \item Parallel (upper energy)
    \item Anti-parallel (lower energy)
\end{itemize}
(Spin is around particle's own axis, not to be confused with angular
momentum). Motions specified by maxwellian velocity distribution, and
collisions dominate the level populations (excite and de-excite).

Energy difference (very small):
\begin{align*}
    h\nu &= 9.4\times10^{-18}\:\mathrm{erg}\\
    \nu &= 1420.4\:\mathrm{MHz}\\
    \lambda &= 21.11\:\mathrm{cm}
\end{align*}

\paragraph{Spontaneous emission:} probability is \emph{very} small:
\[
    A_{kj} = 2.86\times10^{-15}\:\mathrm{sec}^{-1}
    \]
\[
    \rightarrow \mathrm{lifetime} = 1.10\times10^{7}\:\mathrm{years}
    \]
\paragraph{Collisions:}
are more likely to dominate transitions and cause the atoms to flip energy states.
\begin{center}
    H + H $\rightarrow$ H$_{2}^{*}$ $\rightarrow$ H + H
\end{center}
(excited H$_{2}$ molecule - not stable). Chance for collisional excitation
per second: $\gamma n(H)$ [s$^{-1}$].

\begin{center}
    \begin{tabular}{c c}
        $\gamma [\times10^{-11}\;\mathrm{cm}^{3}\mathrm{s}^{-1}]$ & T [K]\\
        \hline
        0.23 & 10\\
        3.0 & 30\\
        9.5 & 100\\
        16 & 300\\
        25 & 1000
    \end{tabular}
\end{center}
As long as $n(H^{o}) > 10^{-2}$ cm$^{-3}$, $\gamma n_{H} >> A$, so
collisions determine excitation and de-excitation\ldots implies
detailed balancing, so Boltzmann distribution is valid for level
populations.
\[
    \frac{n_{1}}{n_{o}} = \frac{g_{1}}{g_{o}}\exp(-h\nu/kT_{k})
    \approx \frac{g_{1}}{g_{o}}(1-\frac{h\nu}{kT_{k}}) \approx \frac{g_{1}}{g_{o}}
    \]
where $T_{k}$ = kinetic gas temperature.


\underline{Derivation}:\mar{35}

Now\mar{36} assume that\ldots

In general, in situations where stimulated emission and absorption can
be neglected, we would obtain
\[
    \frac{b_{1}}{b_{o}} = \frac{1}{1 + \frac{A_{10}}{n_{H}\gamma_{10}}}
    \approx 1
    \]
Note: Potential other excitation process of upper level for HI:
HI + Ly$\alpha$ photon $\rightarrow$ n=2 level ($^{2}P$). $^{2}P$ level might cascade
down to upper $^{2}S$ hyperfine level.

In principle, one might selectively populate the upper level with this
so-called \textit{photon pumping}\footnote{D 17.3}.
It turns out that, due to thermilization of the photons this anomalous
population of the levels does \emph{not} generally occur. (Ly$\alpha$ photons
frequently scatter off A-atom again = LTE situation).\footnote{
    This point is also discussed in Kushami \& Heiles (in Gal. \&
    Extragal. Radio astron., page 100)}

\mar{37}Since the b's are effectively 1, our \textcolor{bred}{main result}
is that
\[
    \frac{n_{1}}{n_{0}}
    = \frac{g_{1}}{g_{0}} \left( 1 - \frac{h\nu}{kT_{k}} \right)
    \approx \frac{g_{1}}{g_{0}} = 3
    \]
(since $h\nu \ll kT$). So $n_{0} = (1/4)n(H^{o}$, $n_{1} = (3/4)n(H^{o}$

Are the HI levels always populated mainly by collisions? No, if $n(H^{o})$ drops
low enough, and if the HI is warm it doesn't work as well.

The warm HI in our Galaxy has $n_{H}\sim0.4$ cm $^{-3}$ and $T_{k} \approx$ 8000 K.
But $T_{s}$ = spin temperature, of order $T_{k}$/5, if it weren't for
excitation by Ly$\alpha$ photons.

But it is true that the 3K background radiation field does \emph{not}
significantly disturb the equilibrium set up by collisions, as we
discussed above.

If $n_{H}$ drops to very low values, collisional excitation is ineffective.
But even in that case, $T_{s} \approx T_{k}$ because of the above mentioned
Ly$\alpha$ excitation.

In any case, in general we now obtain for the emission coefficient:
\[
    j_{\nu} = \frac{3}{4} A \frac{h\nu}{4\pi} n(H^{o}) \phi(\nu)
    \]
which is independent of temperature in most circumstances.
\mynotes{So we don't need temp to know how much HI there is!}
And for the absorption coefficient:
\[
    \kappa_{\nu} = \frac{h\nu}{c} \left[
        n_{0}B_{01} - n_{1}B_{10} \right]
        \phi(\nu)
    \]

\mar{38}
Using the relation between Einstein coefficients:

\textcolor{bred}{Important result:} high $T_{k}$ (and $T_{s}$) and/or
broad $\phi_{\nu}$ both = less absorption. \mynotes{Warm HI has less optical
depth than cold HI for the same column density.}

\subsection{Simple case of a single layer of gas}

\mar{39}
\paragraph{Consider two cases}
\begin{enumerate}[label={(\roman*)}]
    \item $\tau_{\nu_{o}}(L) >> 1$ Optically thick
    \item $\tau_{\nu_{o}}(L) << 1$ Optically thin
\end{enumerate}
In general, the line profile of HI emission is entirely determined by the velocity
of the atoms, so the assumption of a Gaussian profile is correct.

\mar{40}
\subsection{Observing brightness temperature}

\begin{framed}
    Aside: \textbf{How do we observe HI?}
    \begin{enumerate}
        \item Radio telescope
            \begin{itemize}
                \item single dish
                    \begin{itemize}
                        \item ``single beam'': one spectrum per position
                            $1\times1$ pixel!
                        \item Multiple beam system: get 7 = 30 beams (``ish'')
                        \item low angular resolution
                        \item sensitive to all the flux
                    \end{itemize}
                    Make a map by many pointings. Point telescope, take spectra,
                    repeat.
                \item interferometer
                One or $\sim$ few pointings per source. FOV is set by FWHM of beam
                of \emph{one} telescope. Angular resolution set by the longest
                baseline (FOV still the same, regardless of $D$). In one pointing,
                can make a map of an area in the sky. $\theta = \cfrac{1.2\lambda}{D_{max}}$
                \begin{itemize}
                    \item High angular resolution
                    \item Missing baselines in array
                        \begin{itemize}
                            \item No zero-m spacing $\rightarrow$ no net flux
                            \item Missing short spacings $\rightarrow$ negative bowl
                            \item Missing gaps in baselines $\rightarrow$ grating rings
                        \end{itemize}
                        Spectroscopic resolution is generally very good.
                \end{itemize}
            \end{itemize}
    \end{enumerate}
\end{framed}

\subsection{HI emission and absorption}
\mar{41}Distinguish two cases:
\begin{enumerate}
    \item Absorption by an HI cloud of an extended background source, usually
        a continuum source (e.g.\ AGN)
    \item HI self-absorption (enough of it at same velocity)
\end{enumerate}
\paragraph{Case 1:}
Simple picture: the radio continuum source is usually a compact, distant
(extragalactic) radio source (but this does not have to be the case).

Reminder of radiative transfer: suppose that the HI layer has kinetic
temperature $T_{s}$, uniform throughout. Then:
\[
    I_{\nu} = I_{\nu}(0)e^{-\tau_{\nu}}
    + B_{\nu} (T_{s}) \left[ 1 - e^{-\tau_{\nu}} \right]
    \]
Rayleigh-Jeans law:
\[
    B_{\nu} (T_{s}) = \frac{2\nu^{2}kT_{s}}{c^{2}}, \quad h\nu \ll kT_{s}
    \]
Toward the source, we then get:
\[
    T_{b} = T_{bo} e^{-\tau_{\nu}} + T_{s} \left( 1 - e^{-\tau_{\nu}} \right)
    \]
where
\begin{itemize}
    \item $T_{b}$ = observed brightness temperature (as function of $\nu$)
    \item $T_{bo}$ = brightness temperature of background source $\equiv$ $T_{c}$
        (continuum), which is essentially independent of frequency across narrow
        frequency range of 21 cm line. \mynotes{Constant, though technically still
        function of frequency.}
    \item $\tau_{\nu}$ = optical depth ghrough HI layer, at frequency $\nu$.
        Only non-zero right around the 21 cm line; $\nu_{0}$ = 1420.4 MHz.
\end{itemize}

\mar{42}In the direction immediately next to the source we have:
\[
    T_{b, off} = T_{b} e^{-\tau_{\nu}} + T_{s} \left( 1 - e^{-\tau_{\nu}} \right)
    \]
The brightness temperature both on and off the source are for each
individual frequency channel.

\paragraph{Some practical points}
\begin{enumerate}
    \item Observe in a \textit{frequency band} in a number of channels,
        together spanning a range in velocity.
        \[
            \frac{\Delta{v}}{c} = \frac{\Delta\nu}{\nu}
            \]
        Total bandwidth is 100 - 1000 km s$^{-1}$, and the channel spacing,
        $\Delta\nu$ is $\sim$ 1 - 5 km s$^{-1}$.
    \item The assumption here is that the ``off source'' position, the HI
        emission from the cloud is the same as in front of the continuum
        source (so the cloud is rather uniform). Evidently, this assumption
        works better if your radio telescope has high angular resolution.
        $\rightarrow$ need a large (single) dish.
    \item A way to potentially avoid this problem is to have a background
        source that switches on and off (a pulsar). However, in practice,
        pulsars are generally too weak.
    \item Do we really have only one HI cloud along the LOS, and is
        its temperature uniform? We really don't need only one cloud,
        provided the clouds have different \emph{velocities}.
        However, the uniform $T_{s}$ assumption is definitely a problem.
        (More about that later.) We can minimize the complexity in terms of
        many clouds along the LOS by observing clouds and background
        sources at high galactic latitudes.
\end{enumerate}

\mar{43}We observe at each frequency for which there is HI.
Let the
observed brightness temperature at source position after subtraction of
continuum be given by
\[
    T_{b, on} - T_{c} \equiv T_{b, obs}
    \]
We get
\[
    T_{b, off} - T_{b, on} = -T_{c} e^{-\tau_{\nu}}
    \]
\[
    T_{b, off} - T_{b, obs} = T_{c} \left( 1 - e^{-\tau_{\nu}} \right)
    \]
\[
    \left( 1 - e^{-\tau_{\nu}} \right) = \frac{T_{b, off} - T_{b, obs}}{T_{c}}
    \]
Measure (observe) $T_{b, off}$, $T_{b, on}$, and $T_{c}$. Then derive
$\tau_{\nu}$.

Once $\tau_{\nu}$ is known, we can find $T_{s}$ because
\[
    T_{s} = \frac{T_{b,off}}{1 - e^{-\tau_{\nu}}}
    \]
the HI spin or kinetic temperature.

\underline{Reminder of general point}:
\begin{align*}
    T_{b,off} &= T_{s} \left( e^{-\tau_{\nu}} \right)\\
    &= \tau_{\nu}T_{s}\quad \mathrm{for}\; \tau_{\nu} \ll 1\\
    &= T_{s}\quad \mathrm{for}\; \tau_{\nu} \gg 1\\
    T_{b,off} &\leq T_{s}
\end{align*}
The last line is \emph{always} the case!

\paragraph{Discussion of some results}
Many recent references\footnote{
    Dickey et al. ApJ 228, 465 (1979)}\footnote{
    See also Dickey and Lockman, Ann Rev of A \& A 28 (1990)}

\mar{44}Plots

\subsection{Conclusions}
\mar{45}
\begin{enumerate}
    \item The absorption profile $(1 - {e}^{-\tau_{\nu}})$ is always
        sharper and simpler in structure than the emission line profile.
        Follows from Gauss analysis of profiles (so fit Gaussians).
        \[
            T = \sum_{i=1}^{m}{
                T_{i} \exp \left\{ -\frac{1}{2} \left(
                    \frac{v - v_{i}}{\sigma_{i}} \right)^{2}
                \right\}}
            \]
        $\sigma$'s are smaller for absorption profile. However, it is not
        clear as to what extent this is significant, except to note that
        absorption profiles are especially sensitive to cold cloud (cores)
        which presumably are narrower in velocity width.
        ``You have to realize you are comparing apples \& peas'' \mynotes{??}
        Given an HI distribution of column density $N(HI)$ = constant, say
        $\tau T_{s}$ = 20 K (note that $\tau T_{s} \propto N(HI)$).
        Assume that $T_{s}$ = 50, 100, 200 K. Then:
        \begin{center}
            \begin{tabular}{c c c c c}
                $\tau T_{s}$ & $T_{s}$ & $\tau$ & $1-e^{-\tau}$ & $T_{B}^{*}$\\
                \hline
                20 K & 50 K & .4 & 0.33 & 16.5\\
                20 K & 100 K & .2 & 0.18 & 18.0\\
                20 K & 200 K & .1 & 0.095 & 19.0\\
            \end{tabular}
        \end{center}
        * from $T_{B} = T_{s} (1-e^{-\tau})$. So a variation of $\sim$
        factor of 3 in absorption profile, $1-e^{-\tau}$ corresponds to
        variation of only 16\% in $T_{B}$. (Note: what would shape of
        $(1-e^{-\tau})$ and $T_{B}$ look like for single [layer cloud]?)

        Warm HI will only appear in emission, since the optical depth is
        too small to give significant absorption. Only recent deep
        observations begin to see absorption in warm HI.\footnote{
            Look for recent papers by Carl Heiles and by Jay Lockman.}
    \item \mar{46}The variation in spin temperature, $T_{s}$, apparent in
        Fig 3-29, makes onewonder whether there is really one temperature
        in each cloud (there seems to be three in this profile).
        wonder whether there is really only one temperature in the cloud.
    \item The range of $T_{s}$ one finds from absorption line studies is
        50 K $\leq T_{s} \leq$ 1000 K. However, the upper value is only
        a lower limit since we don't see absorption anymore for
        $T_{s} >$ 1000 K. So we have no handle on $T_{s}$ above that.
    \item One can compare $T_{spin}$ with $T_{Doppler}$, the temperature
        derived from the velocity width of each line. It turns out that
        always $T_{spin} < T_{Doppler}$.

        $\rightarrow$ Doppler motions that we see in a cloud
        generally do not correspond to thermal velocities; this is probably due
        to turbulence.

        Analysis of emission and absorption line profiles of HI across our
        Galaxy have led to a continuously evolving pictrue of the HI
        distribution that we have. The early desire to attribute
        all HI to ``clouds'' is understandable, but not always useful.\footnote{
            Good summary of some of the main properties of MW HI
            distribution is given by Mulkami \& Heiles in their 1988 review
            in Gal. \& Extragal. radio astronomy}
\end{enumerate}

\subsection{Some relevant results}
\mar{47}
\begin{enumerate}
    \item HI is \emph{not} concentrated in a small number of giant clouds,
        as H$_{2}$ is; estimates of filling factor range from 20-90\%?!
    \item HI occurs roughly 50/50 in two important forms:
        \begin{itemize}[itemsep=0.5mm]
            \item CNM = cold neutral medium $\sim$ 80 K
            \item WNM = warm neutral medium $\lesssim$ 8000 K\\
                About half of all HI gas, whose emission line
                profiles can be split into two components
                %\footnote{Mebold, 1972}:
                \begin{itemize}
                    \item narrow ($\sigma < 5$ km s$^{-1}$)
                    \item wide ($5 < \sigma < 17$ km s$^{-1}$)
                \end{itemize}
                WNM = \emph{wide} component
                \begin{itemize}
                    \item present in essentially all directions from solar neighborhood
                    \item large vertical scale-geight, possibly up to 480 pc
                \end{itemize}
        \end{itemize}
    \item Clouds are filaments and/or sheets, rather than spheres
    \item At low galactic latitudes, it is not possible to distinguish between
        WNM and CNM because there is too much material along the line of sight.
    \item \textbf{Problem with different temperatures along the line of sight:}
        Illustration: consider just two isothermal clouds. For one cloud:
        \[
            T_{s} = \frac{T_{B}}{1 - e^{-\tau}}
            \]
        For two clouds, we have two spin temperature: $T_{s1} and T_{s2}$.
        \mar{48}Then let $T_{n}(v)$ be the (naively determined) spin
        temperature:
        \[
            T_{n}(v) = \frac
                { T_{s1} ( 1 - e^{-\tau_{1}}) + T_{s2} ( 1 - e^{-\tau_{2}})e^{-\tau_{1}} }
                { (1 - e^{-\tau_{1}-\tau_{2}}) }
            \]
        We see that $T_{n}$ depends on $v$ and ranges from $T_{s1}$ through
        $T_{s2}$, depending on $\tau_{\nu}$. We would conclude that HI exists at
        temperatures intermediate to $T_{s1}$ and $T_{s2}$.
\end{enumerate}

Special cases: \mynotes{compare $\tau_{1}$ to $\tau_{2}$ over velocity range
where $N(HI)$ peaks}:
\begin{enumerate}
    \item $\tau_{1} \gg 1$, $\tau_{2} \ll 1$ $\rightarrow$ $T_{n} \approx T_{s1}$
    \item $\tau_{1} \ll 1$, $\tau_{2} \gg 1$ $\rightarrow$
        $T_{n} \approx T_{s1}\tau_{1} + T_{s2}$. So we increased the cold
        temperature $T_{s2}$ by a non-physical foreground temperature of warm
        gas. Happens a lot in practice!
    \item $\tau_{1} \ll 1$, $\tau_{2} \ll 1$ $\rightarrow$ $T_{n}$ is a
        column density weighted harmonic mean temperature. Example: 80 K CNM +
        8000 K WNM with equal column densities; then $T_{n} \approx$ 160 K,
        so you \emph{overestimate} $T_{cool}$ and \emph{greatly underestimate}
        $T_{warm}$.
\end{enumerate}

\subsection{Some additional points on HI}
\mar{49}
%\footnote{Kulkami \& Heiles}
\begin{itemize}
    \item HI absorption measurements toward Galactic continuum sources provide (some)
        information on the distance to these sources, using the velocity of the HI
        absorption feature to derive a kinematical distance. The usual problem of
        distance ambiguity may still be a problem here as well, although the total
        HI column [density?] that is derived from the absorption measurements helps
        a little in distinguishing near and far distances.
    \item \textbf{Temperature of WNM}\ldots two methods:
        \begin{enumerate}
            \item HI absorption: complicated because $\tau$ is small, plus stray
                radiation. Also biases to low temperatures when cooler gas is
                along the LOS\@.
            \item UV absorption lines: local ($<$ 200 pc) stars give
            $T \gtrsim$ 6000 K for WNM
        \end{enumerate}
        So far: data in agreement with 5000 K $\lesssim T \lesssim$ 8000 K,
        but needs more confirmation.
    \item \textbf{Temperature of CNM}\ldots from absorption spectra.
        We already pointed out correlation of $T_{spin}$ with peaks in
        absorption profile (see fig. in handout, page 44). Interpretation: HI
        clouds are not isothermal blobs. Usually we derive what we call
        $T_{spin}$ of the cloud at the center of the absorption profile, so this
        is $T_{n, min}$
        \mar{50}The absorption lines are \emph{narrow}.
\end{itemize}

Present research on Milky Way HI concentrates on:
\begin{enumerate}
    \item Origin of high velocity clouds
    \item Properties of smallest clouds/features
    \item Search for HI clouds that may contain dark matter, but no stars
\end{enumerate}

\mar{50a, 50b, 50c}
Images and plots.

\mar{I1}
\section{Atomic structure}
\subsection{Introduction}
\subsection{Hydrogen atom \& hydrogen-like atoms (or ions)}
\mar{I2}\underline{Example}: energy of ground states:
\begin{itemize}
    \item H = 13.6 eV
    \item He$^{+}$ = 54.4 eV
    \item Li$^{2+}$ = 122.5 eV
\end{itemize}

\mar{I3}
\mar{I4}
\mar{I5}
\mar{I6}
\subsection{Electron spin}

\mar{I7}But, since $\vec{J}_{1}$ and $\vec{J}_{2}$ can have different directions,
there are different possible values of $\vec{J}$.

\mar{I8}
\subsection{Spin orbit coupling}
Two possible orientations of electron spin with respect to orbital angular momentum
lead to a doubling of energy levels of H-like atoms (except $s$-levels, where
$\ell=0$).

Lines appear in pairs, close together, called doublets.
Example: for Na D lines, $\lambda$ = 5890\AA{}, 5896\AA{}. (Question: are
atoms with filled shells + one electron also hydrogen-like? Doubling is due
to spin-orbit coupling.)

Physically, the spin-orbit coupling produces an extra energy term for the
electron, proportioned to $\vec{S}\cdot\vec{L}$.

\mar{I10}
\subsection{Atoms with multiple electrons}

\mar{I11}
\mar{I12}
\mar{I13}
\mar{I14}
\mar{I15}
\mar{I16}
\mar{I17}
\mar{I18}
\mar{I19}
\subsection{Transition rules}

\mar{I20}
\subsection{X-ray emission}
\subsection{Zeeman effect}

\mar{I21}

\newpage
\mar{51}
\section{HII regions}
\subsection{Introductory remarks}
\underline{Process}: hot OB stars emit UV photons that can ionize the
surrounding neutral H (and He) medium. In practice, this requires stars hotter
than $\sim$ 30,000 K, aka.\ spectral type B0 or earlier.
The physics for planetary nebulae (PN) is similar to that of HII regions,
but central stars in PN may be much hotter and dimmer
because there is ``more stuff''.

UV photons impart energy to gas by ionizing H and He. Excess kinetic energy
of created free electrons is shared with ions and other electrons,
\emph{heating} the medium.

Relevant processes:
\begin{itemize}
    \item photoionization
    \item electron-electron encounters
    \item electron-ion encounters
        \begin{itemize}
            \item Excited ions
            \item Recombination
            \item Bremsstrahlung
        \end{itemize}
\end{itemize}
H atom + photon $h\nu$ $\longrightarrow$ p$^{+}$ + e$^{-}$ + E$_{k}$

\textcolor{bred}{$\Longrightarrow$}
{Since stars create a continuous stream of photons, a
balance is reached between ionization and recombination.}
\footnote{Draine Ch 10, 13.1, 14, 15, 17, 18, 27, 28}
\footnote{Osterbrock AGN$^{2}$}

\mynotes{Field OB stars: were they born there or travel there somehow? Usually
in groups (OB associations), dense clouds, etc.}

\mynotes{Consequences: $\sim 10^{4}$ K. Warm ionized plasma emitting various forms
of radiation: recombination, free-free continuum, bree-bound continuum,
2-photon continuum, collisionally excited forbidden lines from ``metals''.
If there is dust around, get MIR dust continuum. Runaway OB stars (like
teenagers!)}

\mynotes{1 km $\equiv$ 1 pc in $10^{6}$ years. (?)}

\mynotes{$Q_{o}$ [s$^{-1}$]
= number of ionized photons emitted by OB star per second.
Factor of 100 from O 9.5V to O 3V (luminosity classes).}

\subsection{Str\"{o}mgren Theory}
\mar{52}``It's only a model.''
Classical paper\footnote{Str\"{o}mgren 1939 ApJ 89, 526}

\underline{Basic result}:
\begin{itemize}
    \item Hot star in a uniform medium will ionize a spherical volume
        out to a certain radius, whose size is determined by:
        \begin{enumerate}
            \item number of ionizing photons emitted by the star
            \item density of medium (determines recombination rate)
        \end{enumerate}
    \item There is a \emph{sharp} boundary from the ionized to the
        surrounding neutral medium.
\end{itemize}
\subsection{Simple derivation}
Pure H nebula, uniform density. Hot star emits $N_{Lyc}$ photons per
second ($Lyc$ = Lyman continuum), all at the \emph{same} frequency, $\nu_{o}$.
The star will ionize the gas, but there is a balance:
\begin{center}
    \vspace{-2ex}ionization $\longleftrightarrow$ recombination\vspace{-2ex}
\end{center}
Stationary ionization eqilibrium characterized by \textit{degree of ionization}:
\[
    x(r) \equiv \cfrac{n_{e}(r)}{n_{H}}
    \]
where $n_{H}$ = \emph{total}
hydrogen density (ions and neutral H atoms).
$0 \le x \le 1$ ($x = 1 \rightarrow$ complete ionization; $x = 0
\rightarrow$ completely neutral). Problem: what is the shape of $x(r)$?

Important quantity: the Str\"{o}mgren radius, $R_{SO}$, defined as
\[
    \frac{4}{3}\pi\alpha{n_{H}^{2}}{R_{SO}^{3}} \equiv N_{Lyc}
    \]
where $\alpha$ is the recombination coefficient
(collisional process $\sim\rho^{2} \rightarrow$ more encounters, see \ref{alpha}).
The quantity $N_{Lyc}$\footnote{$N_{Lyc} = S_{4}$ in Spitzer's notation.}
gives the total number of recombinations per unit
time (all the magic is here!).\footnote{$S_{4}$ formally should include
contributions from the diffuse radation field.}

\mynotes{or as I like to put it:
\[
    \frac{4}{3}\pi{R_{SO}^{3}} = \frac{ N_{Lyc}}{\alpha{n_{H}^{2}}}
    \]
The volume of the sphere is determined by the number of ionizing photons
divided by the recombination coefficient.}

\subsubsection{Description of photoionization equilibrium}
\mar{53}
\paragraph{Overview:} Consider pure H nebula surrouding a single hot star.
Ionization equilibrium:
\[
    n_{H^{o}}\int_{\nu_{1}}^{\infty}{
        \frac{4\pi{J_{\nu}}}{h\nu}\sigma_{\nu}\mathrm{d}\nu}
    = n_{e}n_{p}\alpha(H,T)
    \]
where
\begin{itemize}
    \item $n_{H^{o}}$ = neutral hydrogen density
    \item $J_{\nu}$ = mean intensity of radiation field
    \item $\sigma_{\nu}$
        $ \approx 6\times10^{-18}\;\mathrm{cm}^{2} $
        = ionization cross-section for H by photons with
        energy above threshold $h\nu_{01}$
    \item $\alpha(H,T)$
        $ \approx 4\times10^{-13}\;\mathrm{cm}^{3}\;\mathrm{sec}^{-1} $
        = recombination coefficient
    \item $n_{e}, n_{p}$ = electron and proton densities
    \item LHS: number of \emph{ionizations} per second per cm$^{3}$
    \item RHS: number of \emph{recombinations} per second per cm$^{3}$
\end{itemize}
To first order, not including radiative transfer effects:
\[
    4\pi{J_{\nu}} = \frac{R^{2}}{r^{2}}\pi{F_{\nu}}(0) =
    \frac{L_{\nu}}{4\pi{r^{2}}}
    \]
(where the far right term is the local radiation field). Order of magnitudes:
\[
    N_{Lyc} = \int_{\nu_{1}}^{\infty}{\frac{L_{\nu}}{h\nu}\mathrm{d}\nu}
    \]
\begin{itemize}
    \item $N_{Lyc} \approx 5\times10^{48}\;\mathrm{sec}^{-1}$ for O6 star
\end{itemize}

\mar{54}Physical conditions: radiative decay from upper levels to $n=1$ is
quick $\rightarrow$ nearly all neutral hydrogen will be in the ground
level. Photoionization takes place from the ground level, and is balanced
by recombination to excited levels, which then quickly de-excite by
emission of photons.

The \emph{photoionization cross-section} ($\sigma \propto \nu^{-3}$) is actually
rather complicated to calcualte. Spitzer (who calls this erroneously the
``absorption coefficient'') gives:
$${
    \sigma_{f\nu} = \frac{7.9\times10^{-18}}{Z^{2}}
    \left(\frac{\nu_{1}}{\nu}\right)^{3}g_{1}f
}$$
where $g_{1}f$ = Gaunt factor (from level 1 to free)\footnote{see, e.g.\
Table 5.1 Spitzer (Draine Ch. 10); see also Fig 13.1 in Draine and Fig. 2.2
in handout (from Osterbrock).}

Since $\sigma_{\nu} \propto \nu^{-3}$, higher energy photons will
typically penetrate further into the nebula before being absorbed.

\subsubsection{The recombination coefficient}\label{alpha}
$\alpha_{n}(H,T)$ = recombination coefficient for \emph{direct} recombination
to level $n$.
$${
    \alpha^{(n)} \equiv \sum_{m=n}^{\infty}{\alpha_{m}}
}$$
where $m$ = other levels the electron cascades through to get to $n$.

\mynotes{Note: Ouside of level 1, higher energy not necessarily better
for ionization. OIII is not necessarily from the same place as OII.
Free electrons: there is a recombination coefficient for every level,
depends on velocity of the electron.}

\mar{55}The summed recombination coefficient ($\alpha$) is the relevant
quantity because every recombination counts, regardless of the level $n$ to
which the electron recombines.

\mynotes{Larger velocities $\rightarrow$ smaller $\sigma$.}

For a distribution $f(v)\mathrm{d}v$, $n_{e}f(v)v$ = number of electrons
with velocity $v$ passing through a unit area per second. Thus
$${
    \alpha_{n} = \int_{o}^{\infty}{v\sigma_{n}(H,v)f(v)\mathrm{d}v}
}$$
where $\sigma_{n}(H,v) \propto v^{-2}$ = recombination cross-section to the level
$n$ for electrons with velocity $v$.
$${
    f(v) = \frac{4}{\sqrt{\pi}}\left(\frac{m}{2kT}
    \right)^{3/2}v^{2}\mathrm{e}^{-mv^{2}/2kT}
}$$
(Maxwellian). Hence, $\alpha_{n} \propto \cfrac{1}{\sqrt{T}}$\footnote{see e.g.\
Spitzer, Table 5.2}

We are interested in the total recombination coefficient to all levels:
However, for direct recombination
to $ n=1 $, a photon is generated which itself can ionize H again; so the
recombination coefficient that we are really interested in is
$\alpha^{(2)} \equiv \alpha_{B}$\footnote{see Draine section 14.2}
\mynotes{This refers to ``case B'': optically thick through Ly$\alpha$
emission. Almost always here! Most common situation. Case A: very low
density gas.}

\mar{56}Neutral fraction \emph{inside} HII region\footnote{see Draine
section 15.3}: Consider pure H nebula, and case B recombination (see
later).
\begin{itemize}
    \item $h\nu$ = \emph{Mean energy} of stellar ionization photons
    \item $\sigma_{pi}$ = photoionization corss-section at freqeuncy $\nu$
    \item $Q(r)$ = rate at which ionizing photons cross spherical surface
    \item $Q_{0}$ = rate at which ionizing photons are emitted by the star
\end{itemize}
In steady state:
\[
    Q(r)
    = Q_{0} - \int_{0}^{r}{
        n_{H}^{2} \alpha_{B} x^{2} 4\pi(r')^{2} \mathrm{d}r'}
    = Q_{0} \left[ 1 - 3\int_{0}^{r/R_{SO}}{
        x^{2}y^{2} \mathrm{d}y} \right]
    \]
here:

\ldots

\mar{57}Rewrite latter as:

so neutral fraction $x$ is small and hence $x^{2} \approx 1$.

HII regions have sharp boundaries (concept of Stromgren radius is useful!)
Solving the equations iteratively actually shows that $Q(r)$ approaches 0
near $ y \sim 1 $ in exponential decay.\footnote{
    See page 59 for shape of $x(r)$ (fig 2.4 and 2.6).}
This is also solved by iteration.

\mar{58}Comments:
\begin{enumerate}
    \item Reality: include He, include diffuse radiation field generated
        from direct recombination to $n=1$ (which can ionize an atom again!),
        realistic stellar energy spectrum below $\lambda = 912$\AA{};
        results are qualitatively the same.
    \item Is assumption of a static ionization equilibrium realistic? No: HII
        region is overpressureized compared to the surrounding medium: $T_{e}$ and
        $n$ higher $\longrightarrow$ HII region expands\footnote{First discussed
        by Kahn (1954)}. Expansion of the ionization front happens rather
        slowly, $v \propto 1-2$ km s$^{-1}$, dynamical timescale.
        $\cfrac{R_{SO}}{v} \sim 10^{7}$ years, of order liftime of HII region.
        So ionization equilibrium does hold.
    \item Recombination time scale = ionization time scale:$${
            t_{ionization} \equiv \frac{\frac{4}{3}(R_{SO})^{3}n_{H}}{Q_{o}} =
            \frac{1}{\alpha_{B}n_{H}} =
            \frac{1.22\times10^{3}}{n_{2}}\;\mathrm{years} =
            t_{recombination}
        }$$
        ($n_{2} = n_{H}$ in units of 100). The two are identical! So
        $t_{recombination} \propto \frac{1}{n_{e}}$
        Key: for $n_{H} > 0.03\;\mathrm{cm}^{-3}\;t_{recombination} <$
        lifetime of massive stars ($\lesssim$ 5 Myr).
    \item As we will see later, each Lyc photon results in $\sim$ 0.35
        H$\alpha$ photons begin produced in this ionization $\rightarrow$
        recombination equilibrium.
    \item Next: influence of dust on $R_{SO}$
\end{enumerate}

\subsection{The spectrum of an HII region}
\mar{63}
\subsubsection{Continuum Radiation}
Sources:
\paragraph{Two-photon decay:}
Transition from $n=2 \rightarrow n=1$ ($\ell=0 \rightarrow \ell=0$) is
strictly forbidden ($\Delta\ell \neq 0$).
The actual process happens in two steps, with a virtual intermediate
phase. Two photons are emitted whose joint energy adds up to the energy
of a Ly$\alpha$ photon = 3/4 ionization energy of H. The two-photon
continuum is symmetric about $\nu_{12}/2$ if expressed in photons
per unit frequency instead.\footnote{See Osterbrock, pp 89-93 for more info.}

\paragraph{Free-free emission:}
Thermal Bremsstrahlung\footnote{
    \S{3.5} Spitzer, Drane ch. 10}:
continuous emission and absoprtion by thermal (Maxwellian velocity
distribution) electrons due to encounters between electrons and positive ions.

\mynotes{Acceleration/deceleration $\longrightarrow$ photons.}

\underline{Classical theory}: electron emits a single narrow Em pulse in time,
with no oscillation in E. $\rightarrow$ FT is broad, almost the FT of a
$\delta$ function. $\rightarrow$ Emission coefficient $j_{\nu}$ is nearly
independent of frequency, up to a \emph{cutoff} frequency, which corresponds to
the Maxwellian velocity distribution of electrons.
\mar{64}So cutoff frequency
is given roughly by $h\nu \sim kT_{e}$. (HII regions: $T_{e} \sim 10^{4}$ K, so
$h\nu \approx 0.87$ eV and $\lambda \approx 1.4 \mu$m).

\underline{Emission coefficient}:
\begin{align*}
    j_{\nu} &= \frac{8}{3}\left(\frac{2\pi}{3}\right)^{1/2}
    \frac{Z_{i}^{2}\mathrm{e}^{6}}{m_{e}^{3/2}c^{3}(kT)^{1/2}}
    g_{ff}n_{e}n_{i}\mathrm{e}^{-h\nu/kT}\\
    &= 5.44\times10^{-39}\frac{g_{ff}Z_{i}^{2}n_{e}n_{i}}{\sqrt{T}}
    \mathrm{e}^{-h\nu/kT}\quad
    [\mathrm{erg}\;\mathrm{cm}^{-3}\;\mathrm{s}^{-1}\;\mathrm{sr}^{-1}\;\mathrm{Hz}^{-1}]
\end{align*}
where $\mathrm{e}^{-h\nu/kT}$ is the exponential cutoff due to $kT_{e}$
and $g_{ff}$ is the Gaunt factor for free-free transitions\footnote{see
Draine, figure 10.1}.

Total amount of energy radiated in free-free transitions per cm$^{3}$
per second:
$${
    \epsilon_{ff} = 4\pi\int{j_{\nu}\mathrm{d}\nu}
}$$
\ldots (long-ass equation)

Remember that the corresponding absorption coefficient $\kappa_{\nu}$ is
related to $j_{\nu}$ by Kirchoff's law, since we deal with \emph{thermal}
emission (\emph{not} blackbody though!) $j_{\nu} = \kappa_{\nu}B_{\nu}(T)$
which leads to [anther long-ass equation].

\paragraph{emission by dust particles}
In practice, dust emission dominates at IR wavelengths, so it is
unlikely to observe the free-free spectrum this far.

\mar{65}

\underline{Important in practical situations}
\begin{enumerate}
    \item Free-free emission is generally observed in the \emph{radio} regime
        for HII regions
    \item For a source at constant $T$:
        \[
            I_{\nu} = S_{\nu} \left( 1 - e^{-\tau_{\nu}} \right)
            \]
        \[
            S_{\nu} = \frac{j_{\nu}}{\kappa_{\nu}} = B_{\nu} (T_{e})
            \]
        \[
            \tau_{\nu} = \int \kappa_{\nu} \mathrm{d}s
            \]
        And if $T_{e}$ is independent of path length, then
        \[
            \tau_{\nu}
            \sim \int n_{e}n_{i} \mathrm{d}s
            \approx \int n_{e}^{2} \mathrm{d}s
            = EM
            \]
        also $\tau_{\nu} \propto \nu^{-2.1} $ (the 0.1 is due to the log term).
\end{enumerate}

\underline{Optically thin emission}

\underline{Optically thick emission}

\mar{66}(Plots)

\mar{67}

\paragraph{Free-bound emission}
\mar{68}

\subsubsection{Line Radiation}
\mar{69}Two types:
\begin{enumerate}
    \item Recombination lines, predominantly from H and He
        \begin{itemize}
            \item optical
            \item radio
            \item forbidden
        \end{itemize}
    \item Collisionally excited lines in heavy elements; these are find-structure
        lines, often forbidden, yet strong. There are recombination lines from
        heavy elements as well, but they are too weak to observe.
\end{enumerate}

\paragraph{Recombination lines:}
\begin{itemize}
    \item emission coefficient:
        $ j_{\nu} = \cfrac{A_{mn'} N_{n} h\nu_{nn'}}{4\pi} \phi(\nu) $
    \item absorption coefficient:
        $ \kappa_{\nu} = \ldots $
\end{itemize}

\mar{71}The calculation is still highly complicated. We can, however,
derive some general properties of $\{N_{n}\}$

\paragraph{Radio recombination lines}
Not very common, lines tend to be very faint.

\mar{72}So:
\[
    \]

\paragraph{Optical and IR recombination lines}
For small $n$, $|E_{n+1} - E_{n}|$ is larger, so collisions become
less important while the $A$ coefficients get bigger.


\paragraph{Collisionally excited lines in heavy elements:}
\mar{79}Generally: electrons don't have enough energy to excite ions from
the ground state to their first excited states. However, many
ions have multiplets in their ground state, due to the coupling of
the electrons individual (orbital and spin) angular momenta.

Many ions have incomplete ``p-shells''; (there is not really a p-shell,
but we talk about electrons which have $\ell = 1$, corresponding to p).
Remember:

\begin{tabular}{l l l l}
    H & 1s\\
    He & 1s$^{2}$\\
    C & 1s$^{2}$ & 2s$^{2}$ & 2p$^{2}$\\
    O & 1s$^{2}$ & 2s$^{2}$ & 2p$^{4}$
\end{tabular}
For $n=2$, there is room for 6 p-electrons, since each $n\ell$ has
a number of $m_{\ell}$ and $m_{s}$ combinations:
\begin{equation*}
    \left.
        \begin{array}{l}
            m_{\ell}: 2\ell+1\\
            m_{s}: \pm \frac{1}{2}\\
        \end{array}
    \right\} 2(2\ell+1)
\end{equation*}
So as long as $\ell=1$ is possible (meaning $n>1$), we have 6 potential
``p-shell'' electrons, \emph{always}.

The enclosed diagram shows the 3 most typical structures of energy
levels for various elements in different ionization stages.
For [OIII] we have, in particular: 2 electrons in ``p-shell'',
$\ell_{1} = \ell_{2}$, so $L = 0, 1, 2$

\mar{81}One can set up 5 equations of statistical equilibrium for each
of these levels. This shows that both the $^{1}S_{o}$ and the
$^{1}D_{2}$ levels are populated essentially exclusively by collisions
from $^{3}P$ level.

So, not surprisingly, the ratio
$${
\frac{N(^{1}S_{0})}{N(^{1}D_{2})}\quad\propto\quad
\exp\left(\cfrac{-\Delta{E}}{kT}\right)
}$$
where $\Delta{E}$ = energy difference between
$^{1}S_{0}$ and $^{1}D_{2}$

Also, observed flux in $\lambda$4363 line is:
$${
    S(4363) \quad\propto\quad \frac{4}{3}\pi{R^{3}}N('S_{0})A_{4363}h\nu
}$$
and similarly for the 4959 and 5007 lines.

The net result, obtained by solving the equations of statistical
equilibrium is:
$${
    \frac{S(4959)+S(5007)}{S(4363)} =
    8.3\exp\left(\frac{3.3\times10^{4}}{T_{e}}\right)
}$$
so this line ratio provides a direct measure of the electron
temperature.

A similar relationship exists for
[NII] (6548, 6583, and 5754) lines.

\underline{In general}: Emission lines arising from ions, such as
O$^{++}$ and N$^{+}$ that have upper energy levels that have considerable
different energies are useful for estimating $T_{e}$. Vice versa,
as we will see, ions with closely spaced upper enrgy levels provie
little info on $T_{e}$, but can be used to derive $n_{e}$
(examples: [SII], [OII] lines).

\underline{Notes about $T_{e}$ determinations}:
\begin{enumerate}
    \item Look at
    \item Collisional
    \item $^{1}D_{2} \;\rightarrow\; ^{3}P_{0} $ transition
\end{enumerate}

\subsection{Types of HII Regions}
\mar{85}In general:
\begin{itemize}
    \item HII regions are associated with molecular clouds and dark
        nebulae.
    \item Structure:
        \begin{itemize}
            \item ``Blister model'' -- cavity inside GMC
            \item ``Champagne model'' -- half cavity at edge of GMC
        \end{itemize}
\end{itemize}
There are also \emph{compact} HII regions, which are only visible at
radio and FIR wavelengths. These are very young objects, often associated
with H2O and OH masers, infrared sources, molecular lines of complex
molecules, etc.

The \emph{range} in properties of HII regions is
enormous\footnote{Kennicut, 1984}.

Modeling of HII region spectra: the probably best known code is
CLOUDY, developed by G. Ferland. You can in put an ioinizing spectrum
from a star or other, a gas distribution, metallicity, etc., and
then calculate the detailed emission line spectrum for the region.

HII regions can be density-bounded or radiation-bounded.
rad: run out of photons before running out of gas.
density: run out of material before using up photons. No clear boundary.

\newpage
\section{Interstellar absorption lines in stellar and quasar spectra}
\mar{89}Discovered in 1904,
\mynotes{before we were aware of dust, or that
MW and M31 were separate galaxies.} Typical optical absorption lines
discovered later include Ca$^{+}$, Na$^{o}$, K, Ti$^{+}$,
and some molecular lines, e.g.\ CH$^{+}$, CH, CN.
Lines from more abundant atoms and molecules (H, H$_{2}$, C, N, O, etc.)
were only found after Copernicus was launched (early 70s) because
$\lambda < 3000$\AA{} for these lines, so they have to be observed
from space.

\underline{Key}: Since the absorption process is proportional to the incoming
intensity, we can find very small column densities of ISM, provided the
background source is strong enough.

Absorption of optical photons generally only occurs from the ground level of
the atom, ion, or molecule, because only these levels are populated in most
conditions (remember that $kT$ = 0.86 eV for T = 10$^{4}$ K, or 8.6 eV for T =
10$^{5}$ K). This implies that in the formation of absorption lines, stimulated
emission from the upper level does not play a role.

\mynotes{O$^{+}$ in HII region sits in ground state, can assume ions
in ISM also stay in $n=1$. Pure absorption foreground screen.
$I = S(\lambda)\left(1-\mathrm{e}^{-\tau_{\lambda}}\right)$, can
leave the exponent out ($\rightarrow 0$).}

This is not the case whenever excitation conditions are such that
absorption lines do arise from excited atoms or molecular levels.
Usually collisions would be responsibe for the excitation, so keep
an eye on $kT$.

\mar{90}Generally, interstellar absorption lines are \emph{narrow} and
\emph{complex} in structure. So we require very high resolution to
resolve them:\\
$\left(\cfrac{\lambda}{\Delta\lambda} > 3\!\times\!10^{5}\right) \Leftarrow$
ideal, but rarely achieved. \mynotes{res: couple thousand}

\begin{framed}%[colback=darkpowderblue!5!white,colframe=darkpowderblue!75!black,title=Aside]
Remember that $\cfrac{\Delta\lambda}{\lambda} = \cfrac{\Delta{v}}{c}$
and for $\cfrac{\lambda}{\Delta\lambda} \ge 3\!\times\!10^{5}
\rightarrow v \le 1$ km per second. Which implies
$\Delta\lambda \sim 0.02$ \AA{} at 6000 \AA{} and
$\Delta\lambda \sim 0.0033$ \AA{} at 1200 \AA{}
\end{framed}

\subsection{Theory of the formation of absorption lines in the ISM}
\mynotes{Ideal background source: bright (especially UV), with a
featureless continuum. $\Rightarrow$ Quasars and AGN or hot stars.
Stars tend to be more complicated, though quasars sometimes have
their own lines as well.}

\subsubsection{Equivalent width}
The high spectral resolution that is required to resolve a line is often
unattainable. However, even if we cannot \emph{resolve} the absorption line, we
can still infer some important properties about the ISM. As long as the
resolution is sufficient to separate absorption lines at different velocities,
coming from different clouds along the line-of-sight, or if only one component
is present, we can use the \emph{area} of an absorption line, or the
\textit{equivalent width}, EW, the width of an absorption line which would
absorb 100\% everywhere, and which would have the same area as the hatched
area. Note that EW is always defined relative to $I_{o}$; we \emph{divide} by
$I_{o}$.

\mar{91}Note also that if we decrease the resolving power of the
spectrograph, EW does not change.

\newpage
\underline{Theoretical expression for EW}:
\[
    \tau_{\nu} = \int_{0}^{\infty}{\kappa_{\nu}\mathrm{d}r};\quad
    \kappa_{\nu} = h\nu_{u\ell}
    \frac{n_{\ell}B_{\ell{u}}-n_{u}B_{u\ell}}{c}\phi(\nu)
    \]
(using ``energy density'' definition of $B$ coefficients).
\begin{itemize}
    \item $u$ = upper level
    \item $\ell$ = lower level.
\end{itemize}

If stimulated emission can be neglected, we have:
\[
    \kappa_{\nu} = \frac{h\nu_{u\ell}B_{\ell{u}}}{c}n_{\ell}\phi(\nu) \equiv
    \frac{\pi\mathrm{e}^{2}}{m_{e}c}f_{\ell{u}}n_{\ell}\phi(\nu)
    \]
\begin{itemize}
    \item $f_{\ell{u}}$ = \emph{oscillator strength}
        \mynotes{(historical terminology)}
    \item $\cfrac{\pi\mathrm{e}^{2}}{m_{e}c}$
        = ``classical cross-section''\footnote{
            see for example Rybicki and Lightman, section 3.6}
        \mynotes{$\sim$ Einstein coefficient.}
\end{itemize}

Use
$\vert\mathrm{d}\nu\vert = \cfrac{c}{\lambda^{2}}\vert\mathrm{d}{\lambda}\vert$
and we obtain:
\begin{align*}
    EW
    &= \int_{o}^{\infty}{
        \left[ 1 - \mathrm{e}^{-\tau_{\nu}} \right] \mathrm{d}\lambda}\\
    &= \frac{\lambda^{2}}{c}\int_{o}^{\infty}{
        \left[1-\mathrm{e}^{-\tau_{\nu}}\right]\mathrm{d}\nu} \\
    &= \frac{\lambda^{2}}{c}\int_{o}^{\infty}{
        \left[ 1 - \exp \left(
        -\int_{0}^{\infty}{
            \left( \frac{\pi\mathrm{e}^{2}}{mc} \right)
            f_{\ell{u}}n_{\ell}\phi(\nu)\mathrm{d}r} \right) \right] \mathrm{d}\nu }
\end{align*}
Now assume: $\phi(\nu)$ independent of $r$ for the particular cloud
that we are observing. Then:
$${
    EW = \frac{\lambda^{2}}{c}\int_{0}^{\infty}{
        \left[ 1 - \exp \left(
        -\frac{\pi\mathrm{e}^{2}}{mc}
        f_{\ell{u}} \phi (\nu) N_{\ell} \right) \right] \mathrm{d}\nu}
}$$
with $N_{\ell} = \int{n_{\ell}\;\mathrm{d}r}$ = \emph{column density}
of atoms in level $\ell$. The quantity $\phi(\nu)N_{\ell}$ is the key
part of this equation.

\mar{92}\underline{Note}: This situation is very different from the
discussion of HI 21-cm abs lines where the upper level \emph{is}
populated, and we had to include stimulated emission to first order.

We next assume a Gaussian velocity distribution (Maxwellian) to
describe the distribution of radial velocities:
\[
    f(v_{rad}) = \frac{1}{\sqrt{\pi}b}\exp  \left[
        - \left( \frac{v_{rad}-v_{o}}{b} \right) ^{2}\right]
    \]
where $b = \sqrt{\cfrac{2kT}{m}}$ (only valid for thermal motions)
and $\phi(\nu)$ becomes:
\[
    \phi(\nu) = \frac{1}{\sqrt{\pi}}\frac{1}{\Delta{v_{b}}}H(a,u)
    \]
\[
    a \equiv \cfrac{\Gamma}{4\pi\Delta{v_{b}}}; \quad
    u = \cfrac{\nu-\nu*}{\Delta\nu_{b}}; \quad
    \Delta\nu_{b} \equiv b\cfrac{\nu_{o}}{c}; \quad
    \nu* = \nu_{o}(1-\cfrac{v_{o}}{c})
    \]
$\nu^{*}$ is the central frequency of the line.
$b^{2} = 2\sigma^{2}$ where $\sigma$ = velocity dispersion, and $a$ is
a measure of where the Gaussian profile levels off to the Lorentzian
wings\footnote{see notes from before, p. 34}
\mynotes{(transition of Gaussian line core to the damping wings)}.

\subsection{The Curve of Growth}
What is all this good for?

For a given spectral line, we can measure $EW$ from observations.
Free parameters: $N_{\ell}$, $b$, and $a$. For a given cloud, $b$
and $a$ are specified, and we can plot $EW$ as a function of $N_{\ell}$.
In practice, one plots $\log(W/\lambda) $ against
$\log ( N_{\ell}f_{\ell{u}}\lambda ) $.\footnote{
    example: figure 3.2 in Spitzer}
This is the \textit{curve of growth}.

We can distinguish three regimes in a curve of growth.
\begin{enumerate}[label={\Roman*}.]
    \item \mar{93}$EW \propto N_{\ell}$ (linear)
    \item $EW$ almost independent of $N_{\ell}$
    \item $EW \propto \sqrt{N_{\ell}}$
\end{enumerate}
Turnover points between the regimes:
\begin{itemize}
    \item Point A: location determined by $N_{\ell}/b$
    \item Point B: location determined by value of $a$
\end{itemize}

\subsubsection{Regimes in the curve of growth}
\begin{enumerate}[label={\Roman*}.]
    \item Here $\tau_{\nu} \ll 1$ everywhere, which implies that all atoms
        see the same incoming continuum intensity.
        $ 1 - e^{-\tau_{\nu}} \approx \tau_{\nu} $ so
        \[
            EW = \frac{\lambda^{2}}{c} \left( \frac{{\pi}e^{2}}{mc} \right)
            f_{{\ell}u} N_{\ell}
            \]
        and
        \[
            \frac{EW}{\lambda} \propto N_{\ell} f_{{\ell}u} \lambda
            \]
    \item Center of line becomes saturated, so absorption reaches 100\%.
        Because the Voigt function has steep edges (not wings!) in the
        central part which is dominated by Doppler motions, $EW$ increases
        only slowly with $N_{\ell}$.
    \item Now if we keep increasing $N_{\ell}$, the optical depth in the wings
        starts playing a role. Here, $EW \propto \sqrt{N_{\ell}}$\footnote{
            See, e.g.\ Mihalas, Stellar Atmospheres, section 10.3}
\end{enumerate}

\subsubsection{Turnover points in the curve of growth}
\mar{94}What determines the location of the turnover points?
\begin{itemize}
    \item Point A (I $\rightarrow$ II): This is where the gas becomes
        optically thick, which is determined by column density $N$ and
        the profile function $\phi(\nu)$, in particular for the line
        center $\phi(\nu*) \propto 1/b$, so $\tau \propto N_{\ell}/b$

        Example: What happens if $b$ increases? $\tau$ becomes smaller
        because the velocity dispersion of the gas is bigger, so the
        line is not saturated as quickly. So the linear part of the
        curve of growth is larger, point A moves toward the upper right.
    \item Point B (II $\rightarrow$ III): This is mainly determined by
        the value of $a$, which specifies where the steep edges of the
        Lorentz profile go over into the extended wings. So essentially
        $\Gamma$, where $\Gamma \sim 1/\tau_{0}$ ($\tau_{0}$ = lifetime
        of the level)
\end{itemize}
\subsubsection{Growth curves in practice}
What do observations tell us?
\begin{enumerate}[itemsep=1ex, label={\Roman*}.]
    \item Suppose you only observe one interstellar line, but you managed
        to resolve it completely. Then you know $ 1 - e^{-\tau_{\nu}} $
        and can find $b$ and $N_{\ell}$.
    \item What if you can't resove the line and can only measure $EW$?
        $EW$ is determined by $N$, $b$, $a$, and $f_{{\ell}u}$ (which is known
        from laboratory experiments).
        \mar{95}Without further knowledge, there is a lot of ambiguity
        since similar values of $EW$ can be produced by different sets of
        $N_{\ell}$, $a$, and $b$. However, if you know that your line is
        optically thin, $a$ and $b$ are not important, and your $EW$
        immediately gives $N_{\ell}$, provided you know $f_{{\ell}u}$.
    \item What if you are not sure that your line is optically thin?
        Generally $a$ is very small, and the damping wings are weak;
        center of line profile is Gaussian, and point A is not influenced
        by the values of $a$, as we saw before.
\end{enumerate}
Suppose now that we have two lines, originating from the same level $E_{j}$,
with equivalent widths $W_{1}$ and $W_{2}$. $N_{\ell}$ and $b$ would be
similar for both. If we know both $f_{{\ell}u}$ values, we can derive
$N_{\ell}$ and $b$.

Examples of such cases:
\begin{itemize}
    \item Ca$^{+}$ H and K lines $\lambda$3934 and $\lambda$3968 from the
        same ground level $^{2}S_{1/2}$
    \item Na$^{o}$ D lines near 5800\AA{} and weak doublet near 3300\AA{}
\end{itemize}

\subsection{UV absorption lines from H and H$_{2}$}
\mar{96}


\section{Dust}
\mar{104}Size range: ``standard large grains'' follow exponential size
distribution, extending from (theoretically) $a_{min} \approx 0.01 \mu$m to
$a_{max} \approx 0.25 \mu$m (huge!)\footnote{Mathis, Rumple, and Nordsieck
(1977)} $n(a) \propto a^{-3.5}$. A grain with size 0.1 $\mu$m = 1000 \AA{}
would contain $\sim 10^{10}$ atoms. Grains are really ``solids''!
\mynotes{(behave like solids, even though tiny.)}

The size distribution was later extended downward to much smaller particles to
account for the excess 12 $\mu$m and 25$\mu$m emission observed by IRAS; this
excess indicated the presence of warm (several hundred K) dust, and it is not
possible to heat large dust grains to such high temperatures (see below). It is
also possible that at the smallest sizes, we have \textit{complex molecules}
rather than grains. A favorite candidate is ``PAH'', or polycyclic aromatic
hydrocarbons, which resemble car soot. The small grains may be as small as 5
\AA{} (0.0005 $\mu$m).

Grains can \emph{scatter} or \emph{absorb} incident UV and optical photons; at
the FIR wavelengths at which they emit, the radiation is optically thin (so
they do not \mar{105} \emph{absorb} FIR photons effectively).

By absorbing a photon, a grain increases its energy and heats up from
$T_{0}$ to $T_{1}$:
\[
    h\nu_{ph} = \frac{4}{3}{\pi}a^{3} \int_{T_{0}}^{T_{1}} {
        C_{V} (T) \mathrm{d}T }
    \]
For small grains, the jump from $T_{0}$ to $T_{1}$ can be very large
(several hundred K), depending on $h\nu_{ph}$, $a$, and $C_{V}$ (the heat
capacity per unit volume). These grains are said to be stochastically
heated to high temperatures. As a result, the grains are found over a wide
range in $T$.\footnote{handout, Fig. 2 from Draine (1990)}
Standard large grains are not much affected by single photons. Instead, they
reach a relatively constant equilibrium temperature when embedded in a
specified radiation field.

\subsection{Absorption efficiency: the Q parameter}
The absorption efficiency of grains is described by the \textit{Q parameter},
defined as:
\[
    Q_{\nu}^{abs}
    = \frac{\sigma_{\nu}}{{\pi}a^{2}}
    \quad
    (= 1 \;\mathrm{for\;a\;blackbody})
    \]
\begin{itemize}[label={}]
    \item $\sigma_{\nu}$ = effective absorption cross-section at frequency
        $\nu$
    \item ${\pi}a^{2}$ = geometrical area
\end{itemize}
\mynotes{$Q$ = 1 for a blackbody.}
In practice, $\sigma_{\nu}^{abs} \sim {\pi}a^{2} $
at UV wavelengths, and $\sigma_{\nu}^{scat} \sim \sigma_{\nu}^{abs}$.
For $\lambda \gg a$ (in IR), $\sigma_{\nu}^{scat} \rightarrow 0$.

\mar{106}The optical depth becomes:
\[
    \tau_{\nu}
    = \int_{R}{ \alpha{\nu} \mathrm{d}r}
    = \int_{R}{ n_{d} \pi a^{2} Q_{\nu} \mathrm{d}r}
    = \int_{R}{ \rho_{d}\kappa_{\nu} \mathrm{d}r}
    \]
\begin{itemize}
    \item $\kappa_{\nu}$ = mass absorption coefficient
    \item $\rho_{d}$ = mass density of a grain
\end{itemize}
We have:
\[
    \kappa_{\nu} = \frac{n_{d}}{\rho_{d}} \pi a^{2} Q_{\nu}
    \]
Typical numbers:
\begin{itemize}[label={}, itemsep=1ex]
    \item $\cfrac{M_{d}}{M_{g}} \sim 0.005 - 0.01$
        (``dust-to-gas'' ratio is $\sim$ 1\%)
    \item $\langle a \rangle \sim 0.1 \mu$m
    \item $\rho_{d} \sim 3$ g cm$^{-3}$
    \item $Q_{100{\mu}m} \sim \frac{1}{644}$ for particular model
\end{itemize}
We can also define the absorption of dust per H-atom:
\[
    \sigma_{\nu}^{H}
    \equiv \frac{\tau_{\nu}}{N_{H}}
    = \frac{1}{X(H)} \frac{M_{d}}{M_{g}} M_{H} \kappa_{\nu} \quad
    [\mathrm{cm}^{2}]
    \]
\begin{itemize}[label={}, itemsep=1ex]
    \item $M_{d} = n_{d}m_{d}$ = dust mass density
    \item $M_{g} = \left( n_{H} + 2n_{H_{2}} \right)
        \cfrac{m_{H}}{X(H)}$ = gas density
    \item $X_{H}$ = fractional mass abundance of hydrogen ($ \sim 0.7 $)
\end{itemize}

\subsection{Calculating dust mass from FIR fluxes}
The volume emission coefficient of dust grains is:
\[
    j_{\nu}
    = n_{d} \pi a^{2} Q_{\nu} B_{\nu} (T_{d})
    = \rho_{d} \kappa_{\nu} B_{\nu} (T_{d})
    = \alpha_{nu} B_{\nu} (T_{d})
    \]
(Kirchoff's law!)

\mar{107}The emission is optically then, so
\[
    I_{\nu}
    = \int_{R}{
        j_{\nu} \mathrm{d}r}
    = \int{
        \rho_{d} \kappa_{\nu} B_{\nu} (T_{d}) \mathrm{d}r}
    \]
\mynotes{$T_{d}$ is measured from $Q_{\nu}B_{\nu}(T_{?})$\ldots ?}

What we measure is flux,
$F_{\nu} = \int I_{\nu} \cos \theta \mathrm{d}\Omega
\approx \int I_{\nu} \mathrm{d}\Omega $ for small $\theta$.

For an optically thin, spherical cloud:
\[
    L_{\nu} = \int_{volume} \epsilon_{\nu} \mathrm{d}V; \quad
    \epsilon_{\nu} = \int j_{\nu} \mathrm{d}\Omega
    \]
\[
    F_{\nu}
    = \frac{1}{4{\pi}D^{2}} \int_{\Omega}{
        \mathrm{d}\Omega \int_{V}{
            \rho_{d} \kappa_{\nu} B_{\nu} (T_{d}) \mathrm{d}V } }
    \]
\[
    \boxed{
    F_{\nu}
    = \frac{M_{d} \kappa_{\nu} B_{\nu} (T_{d})  }{D^{2}}}
    \quad (\star)
    \]
So in practice:
\begin{itemize}
    \item Measure $F_{\nu}$ at $\lambda = \lambda_{1}$
    \item Measure $F_{\nu}$ at $\lambda = \lambda_{2}$
    \item Derive $T_{d}$ from fit to spectrum (see below)
    \item Derive $M_{d}$ from $(\star)$
\end{itemize}
$\Rightarrow$ Temperature variations along the light of sight cause severe
problems\footnote{See handout from Draine (1990).}
\mar{108}But first some more comments on spectra emitted by dust grains.
We saw
\[
    Q_{\lambda}
    = \frac{\sigma_{\lambda}}{{\pi}a^{2}}
    \equiv \frac{A_{n}}{\lambda^{n}}
    \]
\begin{itemize}
    \item $A_{n}$ = constant
    \item $n$ = emissivity law exponent. In practice, $n$ = 1 to 2.
\end{itemize}
Consequence: $\tau_{\lambda} \propto \lambda^{-n}$
(verify). And since emission is optically thin:
\[
    I_{\lambda} \propto \lambda^{-n} B_{\lambda} (T_{d})
    \]
This is called a \textit{modified blackbody spectrum}.
\mynotes{(Greenhouse effect!)}

In terms of frequency and luminosity:
\[
    I_{\nu} \propto \nu^{3+n} \frac{1}{ e^{h\nu/kT} - 1 }
    \propto F_{\nu}
    \]
\[
    L_{IR} \equiv 4{\pi}D^{2} \int{ F_{\nu}\; \mathrm{d}\nu }
    \propto \int_{0}^{\infty}{
        \frac{\nu^{3+n}}{ e^{h\nu/kT} - 1 }\; \mathrm{d}\nu }
    \]
\[
x = \cfrac{h\nu}{kT} \rightarrow \mathrm{d}\nu =
\cfrac{kT}{h} \mathrm{d}x
    \]
\[
    L_{IR}
    \propto \int_{0}^{\infty}{
        T^{3+n} \frac{x^{3+n}}{e^{x}-1}\; \mathrm{d}x T}
    = T^{4+n} \int_{0}^{\infty}{
        \frac{x^{3+n}}{e^{x}-1}\; \mathrm{d}x}
    \]
\[
    \boxed{ L_{IR} \propto T^{4+n} }
    \]
The luminosity rises steeply with dust temperature!
\mar{109}How can we measure the value of $n$?
At long wavelengths, $B_{\nu}(T_{d}) = \cfrac{2kT}{c^{2}} \nu^{2} $
(RJ tail) so $ I_{\nu} \propto \nu^{2+n} $ or
$ I_{\lambda} \propto \lambda^{-(4+n)} $
Unfortunately, measuring the slope of this relation is difficult since we need
measurements in the sub-mm (200 - 1000 $\mu$m), which are tough.
Also, there is no guarantee that the same emissivity law holds over all
wavelengths, since different grain sizes will have different temperatures,
and emit at different wavelengths. But for the standard MRN grains
it should work, if there are no problems of different temperatures
along the line-of-sight.

\newpage
\subsection{Dust temperatures}
If dust grains were bricks \mynotes{(blackbodies)}, they would not
be warm enough to emit in IRAS 60 and 100 $\mu$m bands: the energy
density of the Interstellar Radiation field is comparable to that
of the microwave background, so a brick would reach a temperature of
3-4 K in the ISM. Why are grains warmer? Because they are \emph{not}
blackbody emitters and absorbers, but emit a modified black body
spectrum. One may think of it as a ``greenhouse effect''
\mar{110}operating on microscopic scales: dust is efficient at absorbing
short-wavelength radiation, but inefficient at emitting in the FIR.
Yet, given its temperature, it will emit in FIR. As a result, the grain
is heated above the temperature of a brick before there is equilibrium
between emission and absorption. In essence, the grain traps some heat.

Spitzer: energy gained by a grain heated by radiation:
\[
    G_{r} = c \int_{0}^{\infty}{
        Q_{a} (\lambda) U_{\lambda}\; \mathrm{d}\lambda}
    \]
\begin{itemize}
    \item $G_{r}$ = rate of energy gain per second per unit projected
        area of dust due to absorption of radiation
    \item $c$ = speed of light
    \item $Q_{a}(\lambda)$ = absorption efficiency factor
    \item $U_{\lambda}\; \mathrm{d}\lambda$ = energy density of radiation
        field in the range d$\lambda$
\end{itemize}
The grain emits in the FIR according to a modified Planck law.
\[
    j_{{\nu}E} = n_{d} Q_{a} \sigma_{d} B_{nu} (T_{s})
    \]
\[
    \sigma_{d} = geometrical cross-section
    \]
So $Q_{a}$ is essentially a correction factor to the geometrical cross-section.
In equilibrium: heating = cooling.\mar{111}
\[
    c \int_{0}^{\infty}{
        Q_{a} (\lambda) U_{\lambda}\; \mathrm{d}\lambda }
    = 4\pi \int_{0}^{\infty}{
        Q_{a} (\lambda) B_{\lambda} (T_{s})\; \mathrm{d}\lambda }
    \]
If $Q_{a}(\lambda)$ is independent of $\lambda$, then $T_{s}$ comes out
to be $\sim$ 3.5 K $ (T_{s} = [U/a]^{1/4} ) $ in the ISM.
However, $Q_{a}(\lambda)$ does depend on $\lambda$, going somewhere
as $1/\lambda$ or $1/\lambda^{2}$. This allows $T_{s}$ to increase
substantially, reaching 15 to 18 K.

\subsection{Interstellar extinction}
\mar{116}The presence of dust was demonstrated by Trumpler, who detected the
dimming and reddening of distant stars. \mynotes{High-z proto-galaxies!}

Dust both scatters and absorbs light. The combined effect is
\textbf{extinction}. For a point source (e.g.\ a star), the object is
\emph{dimmed} by extinction, since both scattered and absorbed light do not
reach the observer. For an extended source (e.g.\ a galaxy), some light may be
scatter \emph{into} the line of sight, so the extinction will generally be less
than for point sources. On average, extinction is about 0.6 - 1 mag kpc$^{-1}$
(reddening is about 0.3 mag kpc$^{-1}$ in (B-V) in the plane of the Milky Way -
see \S{}\ref{reddening}).

\mynotes{NGC 6240, a heavily obscured starburst nucleus, emits 10 times more
power in the IR/FIR than in the optical, with peak near 60-100 $\mu$m.
Large z: shifts into sub-mm. Led to detection of Ultra-Luminous
IR Galaxies (ULIRGs), where ``ultra-luminous'' is $> 10^{12}$ L$_{\odot}$.
More typical galaxies have L$_{\mathrm{FIR}} \sim$ L$_{\mathrm{optical}}$.}

\newpage
\subsubsection{The extinction law}
\mar{117}Empirical result: reddening in magnitudes
\[
    r_{\lambda} = a + \frac{b}{\lambda}
    \]
where $a$ and $b$ are roughly constant. This is between two extreme cases:
\begin{enumerate}
    \item grey extinction, $r_{\lambda}$ independent of $\lambda$
    \item Rayleigh scattering, $r_{\lambda} \propto \lambda^{-4}$
\end{enumerate}
$\Rightarrow$ particle size is $\sim 10^{-5}$ cm = 0.1 $\mu$m
\mynotes{(= 100 nm)}.

\[
    \mathrm{d}I_{\lambda}
    = -I_{\lambda}n(x)\sigma_{\lambda}\mathrm{d}\lambda
    = -I\mathrm{d}\tau_{\lambda}
    \]
\begin{itemize}
    \item $n(x)$ = number density of grains at position $x$
    \item $\sigma_{\lambda}$ = extinction cross-section per dust grain at
        wavelength $\lambda$
\end{itemize}
The star is dimmed by
\[
    I_{\lambda} = I_{\lambda} ^{0}\mathrm{e}^{-\tau_{\lambda} }
    \]
where
\[
    \tau_{\lambda}  = \sigma_{\lambda} \int_{0}^{x} {n(x')\mathrm{d}x'
    =\sigma_{\lambda}N(x) }
    \]
In magnitudes:
\begin{align*}
    A_{\lambda}
    &\equiv -2.5\log\left[ \frac{I_{\lambda} }{I_{\lambda}^{0}} \right]\\
    &= -2.5\log\left[ \mathrm{e}^{-\tau_{\lambda}} \right]\\
    &= -2.5\log\left[\mathrm{e}^{\ln(\mathrm{e}^{-\tau_{\lambda}}) }\right]\\
    &= -2.5\ln(\mathrm{e}^{-\tau_{\lambda}}) \log\left[\mathrm{e} \right]\\
    &= -2.5 (-\tau_{\lambda}) \log\left[\mathrm{e} \right]\\
    &= 1.086\tau_{\lambda}
\end{align*}

\paragraph{Dependence on Galactic latitude:}
Assume simplest model of uniform dust layer.

\mar{118}In reality, an exponential z-dependence would be more realistic,
while the dust is also clumped, just like the gaseous ISM.

\subsection{Interstellar reddening}\label{reddening}
Reddening is caused by the wavelength dependence of extinction.
We can define a \textbf{color excess}:
\[
    E( \lambda_{1}-\lambda_{2} )
    \equiv \parunderbrace{
        ( m_{\lambda1}-m_{\lambda2} )}{observed color}
    - \parunderbrace{
        ( m_{\lambda1,0}-m_{\lambda2,0} )}{intrinsic color}
    \]
e.g.:
\[
    E(B-V) = (B-V) - (B-V)_{0}
    \]

\underline{Question}: How is the reddening measured?

Reddening line is defined by the extinction law.
\[
    E(U-B) = \parunderbrace{
        0.72}{``standard Galactic reddening law''}
    E(B-V) +
    \parunderbrace{0.05(B-V)}{minor correction}
    \]
We can define \textit{reddening free colors}, if the shape of the reddening law
is known. For example, the following is reddening free:
\[
    Q \equiv (U-B)
    - \frac{E(U-B)}{E(B-V)}(B-V)
    = (U-B) - 0.72(B-V)
    \]

\mar{119}In general:
\begin{align*}
    m_{i} &= m_{i}^{0} + A(\lambda_{i})\\
    m_{j} &= m_{j}^{0} + A(\lambda_{j})
\end{align*}
\[
    C_{ij} \equiv m_{i} - m_{j}
    = C_{ij}^{0} + E_{ij}
    = C_{ij}^{0} + \left[ A(\lambda_{i}) - A(\lambda_{j}) \right]
    \]
We define
\[
    R_{V} \equiv \frac{A_{V}}{E(B-V)}
    = \frac{\tau_{V}}{\tau_{B} - \tau_{V}}
    \]
For the standard Galactic reddening law, $R_{V} \approx 3.2 \pm 0.2$.

\newpage
The interstellar extinction law is derived from comparing reddened and
unreddened star's spectral energy distribution for stars of the same
spectral type. There are some clever extrapolation tricks involved to
get the zero point. The enclosed table\footnote{
    from Sauage \& Mathis (1979), ARAA17, p 84}
gives a good representation of the standard Galactic curve. Note that
\[
    \frac{A_{\lambda}}{E(B-V)}
    = \frac{E(\lambda-V)}{E(B-V)} + R_{V}
    \]
The broad, smooth curve is not the whole story. There are for example still
several unidentified narrower absorption features which are called the
diffuse interstellar band (DIBs). There are also many line features in the
IR and the 2200\AA{} bump (which is present in the table). Some IR features:
9.7 $\mu$m $\rightarrow$ H$_{2}$O, NH$_{3}$?

\mar{120}One can correlate the hydrogen column densities with dust extinctions
(or optical depths) to find:
\begin{itemize}
    \item N(HI) $\approx 4.8 \times 10^{21} $ cm$^{-2}$ E(B-V)
    \item N(HI + H$_{2}$) = N(HI) + 2N(H$_{2}$)
        $\approx 5.8 \times 10^{21} $ cm$^{-2}$ E(B-V)
\end{itemize}
The scatter in the second relation is considerably less. This relation,
combined with grain theories\footnote{
    e.g. Mathis, Rumple, \& Nordsieck (1977) ApJ 217, 425 or
    Draine \& Lee (1984) ApJ}
produce a dust-to-gas ratio of $\sim$ 1\%.

Origin and composition of grains:\footnote{
    See Draine chapter 23 for more details}
\begin{itemize}
    \item oxygen-rich
    \item carbon-rich
    \item ice coatings?
        \begin{itemize}
            \item smallest grains ($<0.01\mu$m) may be too warm
            \item larger grains, which produce UV and optical extinction,
                may have ice coatings
        \end{itemize}
\end{itemize}
The grain formation and depletion of elements on grains can perhaps be
understood as a condensation sequence (see earlier discussion on
ISM absorption lines, \& Draine figure 23.1).

\newpage
\section{Molecular hydrogen and CO}
\mar{124}Molecular gas is, with HI, the dominant phase of the ISM in galaxies,
in its contribution to the total gas mass. The distribution of H$_{2}$, however,
is very different from HI gas. The gas is much colder and denser, and as a
result, fills only a small part of the volume of the Galactic disk. The gas
is heavily concentrated to the midplane of the galaxy (scale height $\sim$ 60
pc) and is mostly found in Giant Molecular Clouds (GMCs), complexes of small
clumps of molecular gas, gravitationally bound, with overall sizes $\sim$ 50
pc and masses of 10$^{4}$ - 10$^{6}$ M$_{\odot}$. All available evidence
indicates that stars form from molecular gas. There is some diffuse
molecular gas locally at higher latitude, so the scale height cited refers
to the main GMC layer, not all molecular gas.
\mynotes{Forming molecules is not trivial; dust grains may act as a catalyst.}

Since H2 molecule is not
easily observable (see later), we need a tracer. CO is the most abundant
($\sim$ one CO molecule for every 10$^{4}$ H2 molecules). In terms of
observational techniques, the predominant tracer of molecular gas is the
CO molecule, through rotational transitions in the mm wavelength range.
\mynotes{(2.6 mm, to be exact.)}
This sets a very different set of observational challenges compared to the
21-cm line: smoother telescope surfaces, different receivers/noise demands,
smaller primary beams, larger short spacings, problems in interferometer
maps. \mynotes{CO maps are harder to make than HI.}

\subsection{Molecular gas and CO as a tracer}
\mar{128}H$_{2}$ molecule has no permanent dipole moment (because the
center of mass and center of charge coincide). Electronic transitions in the
UV (Lyman and Werner bands) have been detected in absorption, and
rotational-vibrational transitions in IR (H$_{2}$ gas that is warm due to shocks)
but these do not allow general maping of this component.
$\Rightarrow$ Need a different tracer. CO has J=1-0 rotational transition
at 2.6 mm (115.3 GHz) which is \emph{easily excited by collisions with
H$_{2}$} ($h\nu/k$ = 5.5 K) even in cold gas. Critical density for
significant excitation of the 2.6 mm line:
$n_{H_{2}} \geq$ 3000 cm$^{-3}$ if there is a balance between CO-H$_{2}$
collisions and spontaneous decay. However, since lines are optically thick
in reality, the critical density is reduced by $1/\tau$ (where $\tau$ is the
optical depth)\footnote{
    See Draine 19.3.1, 19.3.2}
The result is that $T_{x} = T_{k}$ for $n_{H_{2}} \geq 300$ cm$^{-3}$
(the levels are ``thermalized''). $T_{x}$ is the excitation temperature and
$T_{k}$ is the kinetic temperature.

\subsection{CO luminosity}
\mar{130}
\[
    L_{CO} = 4{\pi}D^{2} \int{ I \mathrm{d}\Omega }
    \]
Cloud of surface area ${\pi}R^{2}$ occupies
\[
    \mathrm{d}\Omega = \cfrac{{\pi}R^{2}}{4{\pi}D^{2}}
    \]
so $L_{CO} \propto {\pi}R^{2} T_{CO} {\Delta}v$
(assumes $T_{CO} \simeq T_{A}^{*}$ \ldots).
For gravitationally bound clouds:
\[
    {\Delta}v = \sqrt{ \frac{GM_{H_{2}}}{R} }
    \]
\[
    \Longrightarrow \quad
    M_{H_{2}} = L_{CO} \sqrt{ \frac{4}{3{\pi}G} } \frac{\sqrt{\rho}}{T_{CO}}
    \]
so
\[
    M_{H_{2}} \propto L_{CO} \frac{\sqrt{\rho}}{T_{CO}}
    \]
$L_{CO}$ is measured (need to know distance). $T_{CO}$ is not necessarily
physical.\footnote{
    See Draine 19.6 and figure on reverse side of handwritten notes.}

\underline{Potential problems}:
How constant is the proportionality? For example, metallicity -
abundance of H$_{2}$ over CO may differ. This is especially a problem
for dwarf galaxies. Note: H$_{2}$ molecules mainly form on grains.
Dissociation energy of H$_{2}$ and CO: 4.48 eV and 11.1 eV.

\underline{CO $\rightarrow$ H$_{2}$ conversion calibration methods}:
\begin{itemize}
    \item optical extinction for Galactic clouds $\leftrightarrow L_{CO}$
        ($A_{V}$).
    \item $\gamma$-ray emission at ``soft $\gamma$-ray energies'' probes
        total gas column.
    \item The famous x-factor! Calibration of $L_{CO}$ to $M_{H_{2}}$.
        Canonical value $x$ (see below).
        Assuming gas-to-dust ratio of 1\%.
    \item virial mass of molecular clouds
\end{itemize}

\underline{Results}:
\[
    \frac{N(H_{2})}{I_{CO}} = 2\!-\!3 \times 10^{20}\quad
    \mathrm{cm}^{-2}\
    (\mathrm{km\ s}^{-1})^{-1}
    \]

\mynotes{Good info about molecular lines and transitions on pages
135 - 141 (photocopy from book)}

\newpage
\section{Hot ionized gas}
\subsection{Collisional excitation and ionization}
\mar{142}We already talked about collisional excitation in:
\begin{description}[labelwidth=5em, leftmargin=8em]
    \item [HI] the resulting level populations are given by Boltzman
        statistics, so we did not have to discuss the basics of the
        collision process
    \item [HII regions] Introduced the rate coefficients, $\gamma$,
        which play a role in the statistical equilibrium equation.
\end{description}
Collisional excitation and ionization are also of crucial importance in hot
gas ($T \sim 10^{5} - 10^{7}$ K). As we saw, photoionization by OB stars
produces HII regions with $T \sim 10^{4}$ K. The temperature of any ISM
phase is determined by the heating and cooling rates, which we discuss
later (see last section).

An O star cannot heat gas to much more than 10$^{4}$ K since most of the
ionizing photons have energies that are only a few eV above the H ionization
potential. More energetic photons ionize He, so again only a few eV is left
(note that 1eV
$\equiv 1.6 \times 10^{-12}$ erg $\equiv kT \rightarrow T \sim 16,000$ K).
Electrons quickly excite forbidden lines which cool the ionized gas to
lower temperatures. Similarly, to collisionally ionize H (13.6 eV) requires
high temperatures (the tail of the Maxwellian velocity distribution
matters). The main mechanism to create hotter gas is by shock ionization
which occurs in supernova remnants, evidently the SN explosion, and any
collision between gas clouds in which velocities \mar{143}in excess of 100
km/s are involved. (Shock models predict a temperature-shock velocity
relationship of the form $T_{s} = 1.38 \times 10^{5} v_{s}^{2}$ K for a
fully ionized gas, where $v_{s}$ = shock velocity in units of 100 km/s).
There is not a lot (in terms of mass) of the hot gas around; it cools very
efficiently by line radiation from collisionally excited lines, some
recombination radiation, and by Bremsstrahlung (which would dominate at
very high T\ldots why?) if density is high.
(What would typical wavelength be at which Bremsstrahlung radiation would
be predominant? \mynotes{x-ray}).
The hot gas is sometimes
referred to as ``coronal'' gas, in similarity to the sun's corona.

\subsubsection{Calculation of collisional rate coefficients}
While we discussed the general effect of collisional excitation on HII
region spectra, we have not in detail discussed how the collisional rate
coefficients are calculated.\footnote{Draine does this in chapter 2.}
Evidently the principal factors to consider are:
\begin{enumerate}
    \item Kinetic energy, or similarly (in collisions of entire clouds) the
        translation of velocity into a temperature regime (see expression
        for $T_{s}$ above).
    \item Probabilities for particular processes (ionization, excitation);
        the rate coefficients need to be determined from momentum physics
        and are then averaged over the relevant velocity
        distribution.\footnote{Draine, 2.1}
\end{enumerate}
Spitzer summarizes ionization by energetic particles as well. This includes
thermal electrons and protons but also cosmic rays.

\subsection{Properties of hot ionized gas and spectrum}
\mar{145}

\newpage
\section{Heating and cooling}\mar{151}
Following Spitzer's general formalism\footnote{
    see also Draine chapters 27, 30, 34}:
\begin{itemize}[label={}]
    \item $\Gamma$ = total kinetic energy \emph{gained} [erg cm$^{-3}$ s$^{-1}$]
    \item $\Lambda$ = total kinetic energy \emph{lost} [erg cm$^{-3}$ s$^{-1}$]
        (aka.\ the ``cooling function'')
\end{itemize}
\mynotes{All the physics are contained in these two quantities; need to find out
what they are.}

For $\Lambda = \Gamma$ we reach equilibrium temperature, $T_{E}$.
Since $\Lambda$ and $\Gamma$ may depend on $T_{E}$ themselves, we can
set up a time dependent differential equation:\[
    \parunderbrace{n\frac{\mathrm{d}}{\mathrm{d}t}\left(\frac{3}{2}kT\right)}
    {rate of increase in thermal energy for a mono atomic gas}
    - \parunderbrace{kT\frac{\mathrm{d}n}{\mathrm{d}t}}
    {work done by gas (energy loss)}
    = \parunderbrace{\sum_{\varrho,\eta}{\left(
    \Gamma_{\varrho,\eta}-\Lambda_{\varrho,\eta}\right)}}
    {Sum over interacting particles}
    =\quad \Gamma - \Lambda
    \]
(Note: in what follows, we generally ignore ``work done by gas'' component.)

For T$\lesssim2\times10^{4}$ K, thermal conduction can be ignored.
If we do have enrgy gain or loss by conduction (electrons), we add
-$\vec{\nabla}\cdot\left(k\vec{\nabla}T\right)$ to the righthand side.
\mynotes{Conductivity is hard to figure out, but not important below 20,000 K.}
($\vec{B}$ fields prevent conductivity across field lines.)

\mynotes{ $PV = nRT = NkT \rightarrow P\Delta{V} $ = work done by gas.
$R = N_{A}k$, where $N_{A}$ is Avagadro's number.}

If a gas is not at its equilibrium temperature (T$_{E}$), we can
define a ``cooling time'' $t_{T}$ \mynotes{(like a half-life, sorta)}:
\[
    \frac{\mathrm{d}}{\mathrm{d}t}\left(\frac{3}{2}kT\right) =
    \frac{3}{2}k\left(\frac{\mathrm{d}T}{\mathrm{d}t} \right) =
    \frac{3}{2}k\left(\frac{T-T_{E}}{t_{T}}\right)
    \]
where $T_{E}$ = constant and $T$ = variable. For $T_{E}$ and $t_{T}$
= constant, $T-T_{E} \propto \mathrm{e}^{-t/t_{T}}$
\mynotes{(exponential decay for cooling). $t$ obviously can't be negative
$\rightarrow$ quasi-stable}

\mar{152}If $t_{T} < 0$, unstable situation and gas will cool or heat
toward an entirely different equilibrium temperature. This is relevant
to the multi-phase ISM. All the physics is in $\Lambda$ and $\Gamma$.

\subsection{General}
\subsubsection{Primary heat source}
The primary heat source is (photo)ionization.
\begin{itemize}[label={}]
    \item $E_{2}$ = kinetic energy of ejected electron
    \item $E_{1}$ = kinetic energy of recaptured electron
\end{itemize}
\mynotes{$\Delta{E} = |E_{2} - E_{1}| \rightarrow$ energy available.
Ionization from the ground state (maximum energy) recombines with
less energy.}
Number of captures to level $j$ of neutral atom:
\[
    n_{e}n_{i}<\omega\sigma_{cj}> \quad [\mathrm{cm}^{-3}\; \mathrm{s}^{-1}]
    \]
where $<\omega\sigma_{cj}>$ = recombination coefficient.
The final net gain associated with electron-ion recombination:
\[
    \Gamma_{ei} = n_{e}n_{i}\sum_{j}{\left(
    \parunderbrace{<\omega\sigma_{cj}>\overline{E}_{2}}{ionization out of $j$} -
    \parunderbrace{<\omega\sigma_{cj}E_{1}>}{recapture back to $j$}
    \right)}
\]
\begin{itemize}[label={}]
    \item $<>$ = average over Maxwellian velocity distribution
    \item $\overline{E}_{2}$ = average over all ionizing photon energies
\end{itemize}
As we have seen, all ionizations take place from the ground level
$\rightarrow \overline{E}_{2}$ is independent of $j$; use recombination
coefficient to all levels $\geq n$:
\[
    \alpha^{(n)} = \sum_{n}^{\infty}{\alpha_{m}}
    \]
and
\[
    \Gamma_{ei} = n_{e}n_{i}\left\{\parunderbrace{
        \alpha\overline{E}_{2}}{$\equiv \alpha^{(1)}$} - \parunderbrace{
        \frac{1}{2}m_{e}\sum{<\omega^{3}\sigma_{cj}>}}{kinetic energy}\right\}
    \]


\paragraph{Main point:}\mar{153}
$\Gamma_{ei}$ is not dependent on ionization probability or radiation density
for \textit{steady state}, where the number of ionizations is equal to the
number of recombinations.

\subsubsection{Primary cooling source}
The primary cooling source is \textbf{inelastic collisions}.
(excitation of energy levels, \mynotes{since particles that undergo
inelastic collisions will lose energy}).
\begin{itemize}
    \item $n_{e}n_{i}\gamma_{jk}$[cm$^{-3}$ s$^{-1}$] =
        number of excitations from level $j \rightarrow k$
        for ions in ionization stage $i$ and excitation state $j$.
    \item $E_{jk} = E_{k}-E_{j}$ = energy lost by colliding electrons
        \mynotes{(in the form of emitted photons).}
\end{itemize}

This cooling is offset by
\textit{de-exciting} collisions, which give energy gain to electrons
\mynotes{(give energy back to the nebula)}.
\mynotes{Net cooling:}
\[
    \rightarrow\Lambda_{ei} = n_{e}\sum_{j<k}{E_{jk}\left(\parunderbrace{
    n_{ij}\gamma_{jk}}{excitation} - \parunderbrace{
    n_{ik}\gamma{kj}}{de-excitation}\right)}
\]
\mynotes{$\gamma$s are collisional rate coefficients.}
\underline{Assumption}:
all photons escape (not true for dense molecular clouds).
Again, in most cases all ions are in the ground level, so we don't
need to sum over $j$, and $n_{ij} = n_{i1} = n_{i}$ \mynotes{(simpler)}

\subsection{HII Regions}\mar{154}
\begin{itemize}
    \item Main heat sources: ionization of H and He
    \item Main cooling source: excitation of C, N, O, Ne
        \begin{itemize}
            \item Heavy elements have low abundance. If they didn't,
                HII regions would cool to very low $T_{E}$!
        \end{itemize}
\end{itemize}

\mynotes{UV = radiation field energy density. Radiation is from
OB stars (ionizing photons) and diffuse radiation in HII regions
from direct recombination to n=1.}

\mynotes{HII Regions: Lyman photons didn't escape; they were all
re-absorbed and turned into Balmer line. Photons absorbed by dust
$\rightarrow$ cooling. Photons scattered $\rightarrow$ \emph{not} cooling.
Keep in mind: \textbf{Do all photons make it out of the nebula?}}

\subsubsection{Heating}
\[
    \overline{E}_{2} = \frac{
        \int_{\nu_{1}}^{\infty}{
            h(\nu-\nu_{1})s_{\nu}U_{\nu}\mathrm{d}\nu/\nu}}{
        \int_{\nu_{1}}^{\infty}{
            s_{\nu}U_{\nu}\mathrm{d}\nu/\nu}}
    \]
\begin{itemize}
    \item $s_{\nu}$ = cross section
    \item $\nu_{1}$ = ionization limit for HI
\end{itemize}

\underline{Problem}:
$U_{\nu}$ is determined by stellar radiation ($U_{s\nu}$) and
diffuse radiation ($U_{D\nu}$), and depends in turn on $n_{H}$, etc.

Spitzer discusses two simple cases:
\begin{enumerate}
    \item Close to exciting star: $U_{s\nu}$ large, $U_{D\nu}$ negligible
        in comparison
    \item Evaluate $\overline{E}_{2}$ for entire HII region.
\end{enumerate}

\underline{Approximation}:
Use dilute blackbody of temperature $T_{c}$ (color temperature)
to describe the stellar radiation field at distances from the stars.

Define: $\psi = \cfrac{\overline{E}_{2}}{kT_{c}}$
\begin{itemize}[label={}]
    \item $\psi_{0} = \psi(r\rightarrow 0)$, so close to star
    \item $<\psi>$ = average of star
    \item $\psi(r)$ = over entire HII region
\end{itemize}

\mar{155}
\[
    \psi_{0}:\quad U_{\nu} = \frac{4 \pi B_{\nu} (T_{c})}{c}
    = \frac{1}{c}\int{I_{\nu}\mathrm{d}\Omega}
    \]
Then it is possible to calculate table 6.1 (Spitzer), which lists values of
$\psi_{0}$ (Draine, table 27.1) for various values of $T_{c}$. \footnote{
    Spitzer also discusses how to calculate $<\psi>$; too lengthy to do here.
}

Result:
\begin{tabular}{c c c c c}
    1.05 & \textless & $<\psi>$ & \textless & 1.65\\
    4000 & \textless & $T_{c}$ & \textless & 64,000\\
\end{tabular}

We had that:
\[
    \Gamma_{ei} = n_{e}n_{i}\left\{\alpha\overline{E}_{2}
    - \frac{1}{2}m_{e}\sum{<\omega^{3}\sigma_{cj}>}\right\}
    \]
where second term = mean energy lost per recombining electron.

We have $\overline{E}_{2}$, now we need
\[
    \sum_{j=k}^{\infty}{<\omega^{3}\sigma_{cj}>}
    = \frac{2A_{r}}{\sqrt{\pi}} \left(\frac{2kT}{m_{e}}\right)^{3/2}
    \beta\chi_{k}(\beta)
    \]
\begin{itemize}[label={}]
    \item $A_{r}$ = ``recapture constant''\footnote{Spitzer, page 100}
    \item $\beta = \cfrac{h\nu_{1}}{kT}$
\end{itemize}
$\chi_{k}(\beta)$ are listed in table 6.2 in Spitzer.
They are ``energy gain functions'' with values from $\sim$0.4 to 4.0.

\subsubsection{Cooling}

Use directly the basic equation using
\[
    \frac{n_{k}}{n_{j}} = \frac{b_{k}}{b_{j}}
    \frac{g_{k}}{g_{j}}\exp\left(-\frac{h\nu_{ju}}{kT}\right)
    \]
and
\[
    \frac{b_{2}}{b_{1}} = \frac{1}{1 + A_{21}/n_{e}\gamma_{21}}
    \]
(discussed for HI emission) for a two-state ion.
3 level ions are more complicated.
\begin{itemize}
    \item $\frac{g_{k}}{g_{j}}\exp\left(-\frac{h\nu_{ju}}{kT}\right)
        \rightarrow$ Boltzmann equation
    \item $\cfrac{b_{2}}{b_{1}} \rightarrow$ ``b correction'', since we
        can't assume Boltzmann level populations
\end{itemize}

\subsubsection{Results}\mar{156}
See figures on handout (page -157-).

\paragraph{Some main points:} two groups of coolants
\begin{enumerate}
    \item meta-stable fine-structure levels in ground.
        Spectroscopic term of various ions.
        \begin{itemize}
            \item E$_{ex} < 0.1$ eV $\rightarrow$ IR radiation
        \end{itemize}
        examples in Fig:
        \begin{center}
            \begin{tabular}{c c c}
                [OIII] & $^{3}P_{0} - ^{3}P_{1}$ & 88.4 $\mu$m\\
                       & $^{3}P_{0} - ^{3}P_{2}$ & 32.7 $\mu$m\\
            \end{tabular}
        \end{center}
        \textbf{Weak $T_{E}$ dependence}
    \item meta-stable other spectroscopic terms with excitation energies
        $\gtrsim$ 1 eV, giving rise to optical and UV lines.
        Strong T$_{e}$ dependence of course! Act as thermostat
        $\rightarrow$ will keep T$_{e} \sim 10,000$ K in HII regions.
\end{enumerate}

\paragraph{Note:}
Figure is for (arbitrary) assumption that O, Ne, N are 80\% singly and
20\% double ionized. H is 0.1\% neutral.
\paragraph{Net effective heating rate}
G-L$_{R}$ in the fig. is what we have been calculating
(where G is photoionization and L$_{R}$ is recombination emission for
H and He).

\paragraph{The intersection between heating and cooling gives $T_{E}$}

\paragraph{Optical depth $\tau_{0}$} in figures refers to \emph{distance}
from star. It is the optical depth at the ionization limit of HI, so
it is proportional to N(H).

\paragraph{Outer parts of nebula are hotter!}
\mar{158}This happens because the photoionization cross-section is
proportional to $\nu^{-3}$, so higher energy photons are absorbed
\emph{later}. $T_{E}$ decreases at first, because $U_{s\nu}$ falls.
Then, beyond $r = 0.6R_{S}$, $T_{E}$ increases and is higher near
$r = R_{0}$. Finally, it decreases again in the transition region
where the ionized fraction drops to zero.

The second figure shows what happens if Ne is large enough that some
excited levels of heavy ions are collisionally de-excited;
$T_{E}$ increases, since cooling is less effective.

$\epsilon_{ff}$, the free-free loss rate, follows directly from
(3-56) in Spitzer; it is not very important.

\paragraph{How fast does T change when $T \neq T_{E}$?}
Close to $T_{E}$,
\[
    t_{T} \approx \frac{2\times10^{4}}{n_{p}} \quad [\mathrm{years}]
    \]
Compare to recombination time:\footnote{Spitzer 6-11}
\[
    t_{r} = \frac{1}{n_{e}\alpha}
    = \frac{1.54\times10^{3}\sqrt{T}}{Z^{2}n_{e}\phi_{2}(\beta)}
    \quad [\mathrm{years}]
    \]
\mynotes{(It would take 10$^{4}$ years for recombination to occur if
the star in the HII region disappeared).}
So for $T \sim 10^{4}$ K, $t_{R} \gtrsim t_{k} \rightarrow$
cooling is faster than recombination. (Also, cooling time is much larger
for $T \gg 10^{4}$ K, as we discussed before.)

\mar{159}Consider the handout (cooling and heating in HII regions, taken from
Osterbrock). What is missing in these diagrams?
\begin{itemize}
    \item Cooling by free-bound and bound-bound HI and He?
        No; since each recombination is balanced by an ionization, these
        photons, even though they may well escape from the nebula, do
        not draw net heat from it. Rather, they draw heat from the star
        itself, not the nebula.\footnote{Draine does not agree? He includes
        a recombination cooling rate in Fig. 27.2, 27.3; it is low.
        He keeps it in heating too though\ldots so okay, fine. Whatever.}
    \item photon $h\nu$ absorbed
        $\rightarrow$
        ionizes atom
        $\rightarrow$
        creates electron with \mynotes{(kinetic)} energy
        $\cfrac{1}{2}mv^{2} = h(\nu - \nu_{0})  $
        $\rightarrow$
        electron thermalizes its energy with ions and electrons and
        sets up temperature $T_{e}$.
    \item However, each electron recombines from energy $\cfrac{1}{2}mv'^{2}$,
        which produces photons with a total energy
        $h\nu' = \cfrac{1}{2}mv^{-2} - h\nu_{0} $
\end{itemize}
The \emph{net} energy (or heat) gain from the photoionization is given by
$ \cfrac{1}{2}mv^{2} - \cfrac{1}{2}mv'^{2} $
(Notice that in general, $|v'| < |v|$).
This is to be balanced against the \emph{cooling}, which is predominantly
due to emission from collisonally excited heavier elements
(in HII regions).

\mynotes{Pure H in the ISM: not many photons around; HII regions are fully
ionized. Gas not generally in both phases, e.g.\ 90\% neutral and 10\% ionized.}

\subsection{HI gas}\mar{160}
(Draine, ch. 30)

Not as simple as one might believe\ldots
\paragraph{Simple considerations}
HI neutral $\rightarrow$ few free electrons to share heat.
Only elements with ionization potential (IP) $<$ 13.6 eV will be
ionized (and dust grains, but more on that later).
However, there is still a dominant cooling line from CII
(IP of C is 11.26 eV). So:
\begin{itemize}
    \item $\cfrac{\Gamma_{ei}}{n_{e}} $ down by factor of $\sim$ 1/2000
    \item $\cfrac{\Lambda_{ei}}{n_{e}} $ similar below $T \sim 1000$ K
\end{itemize}
HI is naturally cool

\paragraph{Problem:}
We observe some very warm HI. General solution may well depend on
$T_{e}$, $n$, chemical composition (i.e.\ depletion)\ldots

We will just list the general players.

\subsubsection{Cooling function}
\begin{minipage}[t]{0.5\textwidth}
Cooling function $\Lambda$:
\begin{itemize}
    \item neutral atoms
    \item ions
    \item molecules
\end{itemize}
\end{minipage}
\begin{minipage}[t]{0.5\textwidth}
Excitation sources:
\begin{itemize}
    \item electrons
    \item H atoms
\end{itemize}
\end{minipage}

\mar{161}Some sources in particular:
\begin{enumerate}
    \item CII and SiII excitation by collisions with H. Problem: \emph{depletion}
    \item Excitation of HI, [exp.?] $n=2$ level in warm HI.
        Generally, $n=2$ is not strongly populated, but if it does happen,
        it will cool:
        \begin{itemize}[itemsep=0ex]
            \item Ly$\alpha$ photon $\rightarrow$ dust $\rightarrow$ IR
            \item $\Lambda_{eH} = 7.3\times10^{-19} n_{e}n(HI)
                \exp(-118,400/T) $ [erg cm$^{-3}$ s$^{-1}$]\\
                ($T = h\nu/k$ = 118,400 for a 1216\AA{} photon).
        \end{itemize}
        \mynotes{Fate of Ly$\alpha$ photons: Scatter until absorbed by dust,
        can't do 2-photon emission, level 1 $\rightarrow$ 2p\ldots or something.}
    \item H$_{2}$ molecules: gain or loss source
        \begin{itemize}
            \item loss: excitation of rotational levels
            \item gain: photon pumping of upper rotational levels, followed by
                collisional de-excitation.
                Also other molecular lines may cool (CO, CN, CH,\ldots)
        \end{itemize}
    \item Collisions with dust grains (can both heat and cool):
        Spitzer's fig. 6-2 (page -172- in notes) shows cooling function,
        including HI and H$^{+}$ range.
        Generally, cooling time:
        \[
            t_{T} \approx \frac{2.4\times10^{5}}{n_{H}} \quad [\mathrm{years}]
            \]
        longer than for HII regions \mynotes{(not many lines available)}.
\end{enumerate}

\subsubsection{Heating function}
Very poorly known. $\Gamma_{ei}$ from ionizing elements such as C is best
known (i.e.\ photo-electric heating).

\mar{162}However, using only CII heating would produce $T_{E}$ = 16 K
(Spitzer); clearly too cool. \mynotes{Adding metals is not enough}

\paragraph{Other potential players}
\begin{enumerate}[label=(\alph*)]
    \item Cosmic ray ionization of H, \mynotes{no radiation, but this
        could do it too}.
    \item Formation of H$_{2}$ molecules on grains.
        This releases 4.48 eV. Goes into:
        \begin{itemize}
            \item heating grain
            \item overcome energy of adsorption to grain surface
            \item excitation of new H$_{2}$ molecule
            \item translational kinetic energy that H$_{2}$ molecule
                gets as it leaves the grain.
        \end{itemize}
    \item Photoelectric emission from grains. What is efficiency, as a
        function of $\lambda$? Clearly, complicated problems; see
        Draine for a discussion.
\end{enumerate}
How can HI become 6000K? Cooling is 10 times more efficient
(from figure on page 172: $\cfrac{\Lambda}{n_{H}}$) than for cool HI.
But, if $n_{H}$ is lower, then:
\begin{itemize}
    \item cosmic ray heating more efficient, but still problematic
    \item also grain - photoelectric heating
\end{itemize}

\subsection{Few comments/additions from Spitzer}
See figure 6.2, which sketches what happens at lower T (below 10$^{4}$K).
Note that Spitzer talks about $\cfrac{\Lambda}{n_{H}^{2}}$

\mar{166}Why are there two HI phases?

\mar{167}The ``destruction'' of the hot phase happens (only) through
cooling of the gas. This cooling depends critically on $T_{e}$ and
$n_{e}$, as we will next explore.

\mar{168}Heating and cooling of hot gas:



\end{document}
