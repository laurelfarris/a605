\documentclass[12pt]{article}
\usepackage[left=1in, top=1in, right=1in, bottom=0.5in]{geometry}
\setlength{\marginparwidth}{0.5in}
\setlength{\parindent}{0em}
\setlength{\parskip}{0.75ex}
\pagenumbering{gobble}
\usepackage{graphicx}
%\usepackage{tasks}
\usepackage{lipsum}
%\usepackage{enumerate}
\usepackage{enumitem}
\usepackage{amsmath}
\usepackage{amssymb}
%\usepackage{mdwlist}
%\usepackage{color}
\usepackage{xcolor}
\usepackage{ragged2e}

\usepackage[symbol]{footmisc}
\usepackage{perpage}
\MakePerPage{footnote}

\usepackage{marginnote}
%\renewcommand*{\raggedrightmarginnote}{\centering}
\renewcommand*{\raggedleftmarginnote}{\centering}
\newcommand{\mar}[1]{\hspace{0pt}\marginpar{-{#1}-}}

\definecolor{bred}{rgb}{0.8, 0.0, 0.0}

\usepackage{titlesec}
%\titleformat{<command>}
%   [<shape>]{<format>}{<label>}{<sep>}{<before-code>}{<after-code>}
\titleformat{\section}%
    [hang]
  {\filcenter\fontsize{16}{18}\selectfont\bfseries} %\filcenter\bfseries\LARGE
  {\hspace{-0.25in}\Roman{section}.\;}        % label%    {\thesection} %{<label>}
  {0pt}     % sep
  {}        % before code
\titleformat{\subsection}%
  {\filcenter\fontsize{14}{16}\selectfont\bfseries} %\filcenter\bfseries\LARGE
  {\arabic{subsection}.\;}
  {0pt}     % sep
  {}        % before code
\titleformat{\subsubsection}%
  {\filcenter\fontsize{13}{15}\selectfont\bfseries\itshape} %\filcenter\bfseries\LARGE
  {\arabic{subsection}.\arabic{subsubsection}\;}        % label%    {\thesection} %{<label>}
  {0pt}     % sep
  {}        % before code

%\titlespacing*{\section}{-0.5in}{0ex}{0ex}
%\titlespacing*{\subsection}{0pt}{0.5ex}{-10ex}

% Section references
%\renewcommand{\thesection}{}
%\renewcommand{\thesubsection}{\arabic{subsection}}
%\renewcommand{\thesubsubsection}{\arabic{subsubsection}}

%\setcounter{secnumdepth}{1}


% Lists
\setitemize{itemsep=-1ex}
\setenumerate{%
    itemsep=-1ex}
\setdescription{%
    itemsep=0ex,
    align=right,}
\renewcommand{\labelitemi}{$\vcenter{\hbox{\tiny$\bullet$}}$}
%\renewcommand{\labelitemi}{{\tiny$\bullet$}}
\definecolor{cadet}{rgb}{0.33, 0.41, 0.47}
\renewcommand{\descriptionlabel}[1]{%
    \ttfamily\textcolor{cadet}{#1}
}
% Verbatim
\usepackage{fancyvrb}  % framebox around verbatim text
\makeatletter
\renewcommand\verbatim@font{\normalfont\small\ttfamily}
\makeatother

\usepackage{listings}
\lstset{% general command to set parameter(s)
    basicstyle=\small, % print whole listing small
    keywordstyle=\color{black}\bfseries\underbar,% underlined bold black keywords % nothing happens
    identifierstyle=,
    commentstyle=\color{white},
    stringstyle=\ttfamily,
    showstringspaces=false % no special string spaces
    }

\usepackage{setspace} % spacing between toc items
\usepackage[toc]{multitoc}
\renewcommand*{\multicolumntoc}{2}
%\setlength{\columnseprule}{0.5pt}

\usepackage{hyperref}
\definecolor{darkpowderblue}{rgb}{0.0, 0.2, 0.6}
\hypersetup{colorlinks=true,
    urlcolor=darkpowderblue,
    linkcolor=black % This may be what links the contents in the first place!
}
\urlstyle{same}

\begin{document}
\tableofcontents\newpage

\reversemarginpar
\section{Introduction}
\mar{6}Discuss general viewpoint: What can happen to an atom or molecule or dust
grain sitting in the ISM?
\begin{itemize}
    \item Can it absorb a photon? Energy levels $\leftrightarrow$ $h\nu$
        (Need cross-sections for dust grains)
    \item Can it collide with other particles? $\rightarrow$ collisional
        rates [s$^{-1}$ cm$^{-3}$]
    \item Is there a magnetic field? Is the particle charged?
    \item Are there cosmic rays? These can penetrate dense gas.
\end{itemize}
Three possible sources of ionization (and excitation):
\begin{enumerate}
    \item photons
    \item collisions
    \item cosmic rays
\end{enumerate}
%\begin{tasks}(3)
%\task photons
%\task collisions
%\task cosmic rays
%\end{tasks}
Typical collisional energies: kinetic energy.
Translate $mv^{2} \rightarrow kT \rightarrow h\nu \rightarrow eV$.

\paragraph{Dust grains}\marginpar{-8-} also occur in the neutral medium, and
probably also in the (warm) ionized medium. Dust grains play an important
role in various processes:
\begin{itemize}
    \item extinction of starlight
    \item emission of absorbed energy in FIR
    \item formation of molecules often occurs on grain surfaces
    \item absorption of ionizing UV radiation and Ly$\alpha$ photons
        (reducing amount of ionizing radiation)
    \item heating of HI gas by \emph{photoelectric} emission
\end{itemize}
Composition: carbon and silicates. Typical sizes: 0.01 - 0.1 $\mu$m
(How do we know? $\rightarrow$ shape of extinction curve). Grains as small
as $\sim$ 60 atoms across discovered; evidence from emission lines in NIR
and excess emission at 5 - 40 $\mu$m over what is expected from dust in the ISM\@.
The larger dust grains have temperatures between 10 and 40 K, while the small
ones can be heated to higher temps due to the absorption of even a single
photon (smaller heat capacity, as volume $\propto r^{3}$). A promising
candidate for small dust grains: polycyclic aromatic hydrocarbons
($\sim$ car soot!)

In hot environments, dust grains may be destroyed by \textit{sputtering},
where collisiosn of grains with other atoms, electrons, or molecules
knock molecules off the grains. \marginpar{-9-}At low temperatures,
molecules stick to dust grains, causing depletion of heavy elements
along certain lines of sight (most dust in the plane).
Dust contributes about 1\% of the mass of the ISM in the solar
neighborhood, mostly in the form of large grains.

\paragraph{Ionized gas}
\begin{enumerate}[label=\alph*)]
    \item Photoionization: especially effective near hot stars. Shock
        ionization
    \item Cosmic rays: can occur throughout most of the ISM, so can also
        produce a small amount of ionization in denser gas (though
        recombination happens quickly, so not much of gas is in an
        ionized state at any given time).
    \item Collisions\marginpar{-11-} Hot ($\sim 10^{7}$ K) expanding bubble
        sweeps up shell of warm ($\sim 10^{4}$ K) ionized gas, which shows
        a different optical spectrum than HII regions. The hot gas shows
        up by:
        \begin{itemize}
            \item Free-free and line x-rays
            \item absorption lines of highly ionized species toward
                bright UV sources
        \end{itemize}
        Kirchoff's law: apparently can still see absorption lines when looking
        through a gas that is hotter than the source behind it!
\end{enumerate}

\paragraph{Magnetic fields and cosmic rays}\marginpar{-13-}
In the solar neighborhood, $\mathbf{B} \sim 2-5\times10^{6}$ Gauss.
This follows from measurements of \textit{Faraday rotation}, giving
$<n_{e}B_{||}>$ toward pulsars and radio sources. The random component of
the $B$ field is probably as large as the uniform component. In clouds,
the $B$ field can be much higher, $\sim 70 \mu$G (from Zeeman effect splitting
measurements).

The magnetic field is important for several reasons:
\begin{enumerate}
    \item It aligns elongated grains, giving rise to polarization
        of starlight
    \item Causes relativisitic electrons to emit synchrotron radiation,
        and most likely plays a role in accelerating electrons to
        relativistic velocities (``magnetic bottle'', Fermi acceleration).
    \item Provides pressure support against gravitational collapse of matter
        since it is frozen into the matter due to ionization heavy elements.
        It also seems to play animportant role in solving the angular momentum
        problem in star formation.\marginpar{Shu et al.}
\end{enumerate}

Total energy density of cosmic rays in solar neighborhood:
$U_{R} \sim 1.3\times10^{-12}$ erg cm$^{-3}$.
Why are cosmic rays important?\marginpar{-14-}
\begin{itemize}
    \item They produce $\gamma$-rays through collisions with atoms and
        molecules. The observed $\gamma$-ray intensity from the ISM forms
        an excellent independent measure of the total amount of matter
        between stars. For example, calibrating the conversion factor of
        CO line intensity to H$_{2}$ mass.
    \item Provide pressure against gravitational collapse
\end{itemize}
Five pressures that play an important role in supporting the ISM against
gravitational collapse:
\begin{enumerate}
    \item thermal $P=nkT$
    \item magnetic $P=\cfrac{B^{2}}{8\pi}$
    \item turbulent (bulk motion)*
    \item cosmic ray
    \item radiation
\end{enumerate}
* The cloud to cloud velocity dispersion due to turbulence on various
scales increases line widths [over?] thermal widths. From the table
on page -12-, you will note rough thermal pressure euqilibrium between
the components. This is not a coincidence; in fact, some of this
information was inferred by assuming pressure equilibrium. The
argument is that if there were no equilibrium, the resulting
perturbations would be wiped out on sound-crossing time scales, which
are short compared to\marginpar{-15-} the time scales we wouuld
consider the ISM to evolve over.

However, the actual evidence for equilibrium in the thermal pressure
is scarse, and there are claims that it is not true in the very local
ISM\footnote{\href{http://www.nature.com/nature/journal/v375/n6528/abs/375212a0.html}
{Bowyer et al. \textit{Nature} 1905}}.

There seems to be a ``cosmic conspiracy'': the estimates for the
thermal, magnetic, and cosmic ray pressure for the solar neighborhood
give roughly equal numbers for all three. Thus it may be inappropriate
to only consider the thermal pressure (the only one that can be
measured with much certainty). Interestingly, the magnetic pressure
number is also very similar to the energy density of the CMB\@.
\footnote{Draine discusses possible reasons in section 1.3}
\footnote{Supplemental info in Draine chapter 1.}

Next\marginpar{-16-} section is sort of a shortened condensation of Draine's chapters
2 and 3. We may come back to specific topics discussed there in more
detail.
\section{Validity of the laws of statistical physics in ISM conditions}
Four major laws of statistical physics:
\begin{enumerate}
    \item \textbf{Maxwellian} velocity distribution
    \item \textbf{Boltzmann distribution} of energy levels in atoms and molecules
    \item \textbf{Saha equation} for ionization equilibrium
    \item \textbf{Planck function} for radiation
\end{enumerate}

\paragraph{Maxwellian} $T$ is defined by motion of particles.
($\vec{\omega}$ = velocity = $\vec{v}$ in Draine).
$f(\vec{\omega})\mathrm{d}\vec{\omega}$ = fractional number of
particles whose velocity lies within the three-dimensional volume element
$\mathrm{d}\vec{\omega} =
\mathrm{d}\omega_{x}\mathrm{d}\omega_{y}\mathrm{d}\omega_{z}$,
centered at velocity $\vec{\omega}$.

In thermodynamic equilibrium, $f(\vec{\omega})$ is isotropic, so
$\vec{\omega} \rightarrow \omega$.
$${ f(\omega) =
    \frac{\ell^{3/2}}{\pi^{3/2}}\exp(-\ell^{2}\omega^{2});\quad
    \ell^{2} = \frac{m}{2kT} = \frac{3}{2<\omega^{2}>}
}$$ $${
    f(\omega) = \left(\frac{m}{2\pi kT}\right)^{3/2}
    \exp\left(-\frac{m\omega^{2}}{2kT}\right)
}$$
For two groups of particles with different masses, we replace $\omega$
by $u$, the relative velocity between the two groups, and $m$ by the
reduced mass $m_{r} = \cfrac{m_{1}m_{2}}{m_{1}+m_{2}}$.
For\marginpar{-17-} H atoms colliding with particles of mass $Am_{H}$,
Spitzer derives:$${
    <u> = \left[\frac{8kT}{\pi m_{r}}\right]^{1/2}
    = 1.46\times10^{4}\sqrt{T}\left(1+\frac{1}{A}\right)^{1/2}\;
    [\mathrm{cm\; s}^{-1}]
}$$
\ldots Speed vs.\ velocity

The Maxwell velocity distribution is characterized by several speeds:
\begin{itemize}[itemsep=1ex]
    \item Most probable speed: $\omega_{o} = \sqrt{\cfrac{2kT}{m}}$
    \item RMS speed: $<\omega^{2}>^{1/2} = \sqrt{\cfrac{3kT}{m}}$
    \item RMS velocity in one direction:
        $<\omega_{x}^{2}>^{1/2} = \sqrt{\cfrac{kT}{m}}$
\end{itemize}
\paragraph{Boltzmann distribution}
gives the
\marginpar{-18-}
population of energy levels in an atom or molecule:$${
    \frac{n_{u}}{n_{l}} =
    \frac{g_{u}}{g_{l}}\exp\left[-\left(E_{u}-E_{l}\right)/kT\right] =
    \frac{g_{u}}{g_{l}}\exp\left[-h\nu_{o}/kT\right]
}$$
where $n_{u,l}$ are the number densities, $g_{u,l}$ are the
statistical weights, and $E_{u,l}$ are the energies of the levels.
(Partition function: summing over all energy levels\ldots chemical
term).

\paragraph{Saha equation} describes ionization equilibrium:$${
    \frac{n_{i+1}}{n_{i}} = \frac{g_{i+1}g_{e}}{g_{i}}
    \left(\frac{2\pi m_{e}kT}{h^{2}}\right)^{3/2}
    \exp\left(-I/kt\right)
}$$
where $I$ is the ionization potential for an ion in the ground state
and initial ionization state $i$ (aka, the energy required to ionize
from $i$ to $i+1$). $g_{e} = 2$ (two spin conditions). Electrons
affect whether and how easily atoms can be ionized.

\paragraph{Planck function} specifies the radiation field:
\footnote{
    $e^{x} \approx 1+x$ for $x<<1$\\
    $e^{h\nu/kT}-1 \approx 1$ for $h\nu<<kT$\\
    $e^{h\nu/kT}-1 \approx e^{h\nu/kT}$ for $h\nu>>kT$}
\begin{align*}
    B(\nu) &= \;\frac{2h\nu^{3}}{c^{2}}\frac{1}{\exp[h\nu/kT]-1}\\
    &= \;\sim \frac{2\nu^{2}}{c^{2}}kT\quad \mathrm{for}\;h\nu<<kT\;
    (\mathrm{Rayleigh-Jeans})\\
    &= \;\sim \frac{2h\nu^{3}}{c^{2}}\exp[-h\nu/kT]\quad\mathrm{for}\;h\nu>>kT
    (\mathrm{Wien})
\end{align*}

\section{Statistical equilibrium}
\textcolor{bred}{The\marginpar{-19-} four laws discussed above hold under
thermodynamic equilibrium (TE).}
However, this is not often the case for the ISM\@.
Thermodynamic equilibrium requires \textbf{detailed balancing},
i.e.\ each process is as likely to occur as its inverse.
\underline{Example}: Consider the 3727 \AA{} emission from O$^{+}$. This is
a forbidden transition (actually a doublet). The excitation of the electron
level occurs through collisions with electrons, in most conditions in the
ISM\@. If detailed balancing were to hold, de-excitation should also occur
by collisions. However, as we will see, under the low density conditions
found in the ISM, collisions are rare, and de-excitation is more likely to
proceed through emission of a photon, in spite of the fact that we are
dealing with a forbidden transition. Thus [OII] emission can be quite strong,
and by converting collisional (kinetic) energy into radiation, we actually have
created a cooling mechanism for the gas.

Another reason why TE does not hold is the strong \textbf{dilution} of
the radiation field: the concept of a dilute radiation field is quite familiar.
For example, the sun's photosphere is $\sim$ 6000 K, and at the surface the
flux leaving the sun is approximately that of a blackbody of this temperature.
However, the Earth is not 6000 K because by the time the radiation reaches us,
it is diluted. \textcolor{bred}{A diluted radiation field is one in which
the energy density does not match the color temperature.}

For \marginpar{-20-}the solar neighborhood, the total energy density of the
radiation field due to all stars in that volume is about 1 eV cm$^{-3}$
(close to cosmic ray density, as mentioned before). When interpreted as an
average temperature using the Stefan-Boltzmann law (energy density of a
blackbody, u = aT$^{4}$), this energy density implies an equivalent temperature
of $\sim$ 3 K. Yet the color temperature implied by the shape of the spectrum
of this Interstellar Radiation Field (ISRF) is that of A and B stars
(T $\sim$ 10$^{4}$ K). So there is a \textbf{dilution factor} $W$ given by:
$${ W \approx (\frac{3}{10^{4}})^{4} \approx 0.25\times10^{-14}
}$$
We conclude that using the Planck law to describe intensities is not correct.

What about the other laws?

\paragraph{1. Maxwell velocity distribution} Good news! It is generally
valid. Detailed balancing is possible for the elastic collisions that are
generally occurring. Because the maxwellian distribution is a good
description of the motions of the particles, we can define a kinetic
temperature which describes the physical condition of the gas. Often,
for a plasma, the kinetic temperature is equal to the electron temperature:
$T_{ions} = T_{e}$. $T_{ions} \neq T_{e}$ may occur behind shocks.

\paragraph{2. Boltzmann distribution} is rarely correct.\marginpar{-21-}
If excitation and de-excitation occurs by photons, we may still not have a
Boltzmann distribution because the photon distribution is not given by the
Planck function. Often we do not even have detailed balancing.
However, as we will see, sometimes the distribution of excited levels
is not too different from Boltzmann distribution. This happens when collisions
dominate excitation and de-excitation, while radiation is relatively unimportant.

To describe situations close to TE, Spitzer introduced the so-called
b-factors (Draine calls them ``departure coefficients'').
$${ b_{j} \equiv
    \frac{n_{j}(\mathrm{true\; distribution}}{n_{j}(\mathrm{LTE\; distribution})}
}$$
Example: in an HII region, the hightest excited levels of HI have $b_{j} \sim 1$.
Some radiation does escape (producing radio recombination lines) but collisions
dominate the level populations. Since motions of particles \emph{are} described
by a Maxwellian velocity distribution, whenever collisions dominate the level
population they will closely follow a Boltzmann law.

In \marginpar{-22-}general, a Maxwellian velocity distribution tends to set up
a Boltzmann population for energy levels in the atoms/particles \emph{if}
transitions resulting from emission and absorption of photons are relatively
unimportant, and collisional (de-)excitation is dominant.
In the case of the highly excited levels in H mentioned before,
collisions with electrons are dominant.

\paragraph{3. Saha equation} is generally not valid. There is no detailed
balancing, and even though the ionization and recombination processes are
each other's inverse, the ionization process is determined by the photon
field in most cases, while the recombination process is determined by collisions
between A$^{+}$ and e$^{-}$. The collision rate depends on \{$n_{e}, n_{A^{+}}$\}
and $T_{e}$, but the ionization is dependent on $T_{radiation} (\neq T_{e})$.

So in general, assume \textbf{statistical equilibrium}, where there is
a balance between transitions one way and the other way, no matter
what process caused each transition.

In level $i$, we have $n_{i}$ atoms cm$^{-3}$, and
$R_{ij}$ is the rate coefficient such that

$n_{i}R_{ij}$ [s$^{-1}$] = \# transitions from level $i$ to level $j$\\
$R_{ji}n_{j}$ [s$^{-1}$] = \# transitions from level $j$ to level $i$\\

$${ \frac{\mathrm{d}n_{i}}{\mathrm{d}t} =
    \sum_{j}\left(-R_{ij}n_{i} + R_{ji}n_{j}\right);\quad i=1,2,\ldots
}$$
In\marginpar{-23-} statistical equilibrium,
$\cfrac{\mathrm{d}n_{i}}{\mathrm{d}t} = 0$. The rate factor $R_{ij}$
includes \emph{all} possible processes that would take the atom or
molecule from level $i$ to $j$. In the worst case, you would have to
include many processes to calculate the $n_{i}$ values. This requires
knowledge of a lot of physical input parameters, e.g.\ cross sections
for particular processes, collisional rate coefficients
\footnote{Drane section 2.1}, etc.

In other cases, where only one or two processes matter, the situation
can be very simple. We will encounter cases of each.

Before going more into Ch 2 and 3 in Draine, we will first discuss
some basic radiative transfer.\footnote{RL Ch 1, Draine Ch 6, 7}

\section{Radiative Transfer}\marginpar{-24-}
\begin{minipage}{0.45\textwidth}
    \underline{Flux at surface of a sphere}:
    \begin{list}{}
    \item $F_{\nu} = \pi{B_{\nu}}$ for blackbody
    \item $F_{\nu} = \pi{I_{\nu}}$ for isotropic emitting non-blackbody
    \end{list}
\end{minipage}
\hfill\begin{minipage}{0.45\textwidth}
    \underline{Flux at a distance $r$}:
    \begin{list}{}
    \item $F_{\nu}(r) = \pi{I_{\nu}}(\frac{R}{r})^{2}
        = \frac{L_{\nu}}{4\pi{r^{2}}}   $\\
        where $R$ = radius of body and $L_{\nu}$ = luminosity of body
        [erg s$^{-1}$ Hz$^{-1}$]
    \end{list}
\end{minipage}

\marginpar{-25-}
\subsection{Energy density of radiation}
$u_{\nu} = \frac{1}{c}\int{I_{\nu}\mathrm{d}\Omega}$ [erg cm$^{-3}$
Hz$^{-1}$]
\subsection{Radiation pressure} $P_{\nu} =
\frac{1}{c}\int{I_{\nu}\cos^{2}\theta\mathrm{d}\Omega}$

\subsection{Emission and absorption coefficients}
\subsubsection{Emission}
\begin{itemize}
    \item $j_{\nu}$ [erg cm$^{-3}$ sec$^{-1}$ $\Omega^{-1}$ Hz$^{-1}$]
        spontaneous emission coefficient
    \item $[j_{\nu}]$ = volume emission coefficient
    \item $j'_{\nu} = \cfrac{j_{\nu}}{\rho}$ [erg g$^{-1}$ sec$^{-1}$ $\Omega^{-1}$ Hz$^{-1}$]
        = mass emission coefficient
    \item $j_{\nu} = \cfrac{\epsilon_{\nu}}{4\pi}$ (isotropic emitter);
        $\epsilon_{\nu}$ = emissivity [erg cm$^{-3}$ s$^{-1}$ Hz$^{-1}$]
\end{itemize}
\subsubsection{Absorption}\marginpar{-26-}
\begin{itemize}
    \item $\kappa_{\nu}$ [cm$^{-1}$] volume absorption coefficient (includes
        \emph{stimulated emission}, aka ``negative absorption'').
    \item $\kappa'_{\nu} = \cfrac{\kappa_{\nu}}{\rho}$ [g$^{-1}$ cm$^{2}$]
        = mass emission coefficient
    \item microscopically: $\kappa_{\nu} = n\sigma_{\nu}$
\end{itemize}
Loss of intensity in a beam of light as it travels distance $ds$:$${
    dI_{\nu} = -\kappa_{\nu}I_{\nu}ds
}$$
The absorption process refers to the sum of ``true absorption'' + stimulated emission,
($\sim$ net emission). It
removes a \emph{fraction} of the incoming radiation.

\paragraph{mean free path: $\ell_{\nu}$} $\tau_{\nu} = \kappa_{\nu}\cdot\ell_{\nu}$,
so for condition $\tau=1$, $\ell_{\nu} = \cfrac{1}{\kappa_{\nu}} = \cfrac{1}{n\sigma_{\nu}}$

\section{Radiative transfer equation}$${
    \frac{dI_{\nu}}{ds} = -\kappa_{\nu}I_{\nu} + j_{\nu}
}$$
Define \underline{optical depth} (unitless): $${
    \tau_{\nu} = \int{\kappa_{\nu}ds} }$$
then $${
    d\tau_{\nu} = \kappa_{\nu}ds }$$
or $${
    \cfrac{dI_{\nu}}{d\tau_{\nu}} =
    -I_{\nu} + \cfrac{j_{\nu}}{\kappa_{\nu}}
    \equiv S_{\nu}}$$
where $S_{\nu}$ = \underline{source function} 
\begin{itemize}
    \item $\tau \gtrsim 1$ optically \emph{thick} emission
    \item $\tau < 1$ optically \emph{thin} emission
\end{itemize}

\paragraph{Formal solution:}$${
    I_{\nu}(\tau_{\nu}) = I_{\nu}(o)\mathrm{e}^{-\tau_{\nu}} +
        \int_{0}^{\tau_{\nu}}\!{\mathrm{e}^{-(\tau_{\nu}-\tau'_{\nu})}
        S(\tau'_{\nu})\mathrm{d}\tau'_{\nu}}
}$$
$\rightarrow$\marginpar{-27-}attenuated incoming beam + contribution from gas itself.

\paragraph{Special cases}
\begin{enumerate}
    \item Source function is constant throughout source: $ I_{\nu}(\tau_{\nu}) =
        I_{\nu}(o)\mathrm{e}^{\tau_{\nu}} + S_{\nu}(1-\mathrm{e}^{-\tau_{\nu}})$
        \begin{itemize}[itemsep=1ex]
            \item optically thick emission:
                $I_{\nu}=S_{\nu}$
            \item optically thin emission:
                $I_{\nu}=I_{\nu}(o)(1-\tau_{\nu}) + \tau_{\nu}S_{\nu}$
        \end{itemize}
    \item Thermal radiation: $S_{\nu} = B_{\nu}(T) $ (The Planck function)
        \begin{itemize}[itemsep=1ex]
            \item optically thick emission:
                $I_{\nu}=B_{\nu}$
            \item optically thin emission:
                $I_{\nu}=\tau_{\nu}B_{\nu}$
        \end{itemize}
\end{enumerate}

For \emph{radio} emission: $I_{\nu}$ is replaced by the \underline{brightness temperature},
$T_{b}$, defined as:$${
    I_{\nu,\mathrm{obs}} \equiv B_{\nu}(T_b)
}$$
In the Rayleigh-Jeans limit for $B_{\nu}$ we get:$${
    I_{\nu,\mathrm{obs}} = \frac{2\nu^{2}kT_{b}}{c^{2}}
}$$ or $${
    I_{b} = \frac{I_{\nu}c^{2}}{2\nu^{2}k}
}$$
Solution of the transfer equation in terms of $T_{b}$:$${
    T_{b,\mathrm{obs}} = T_{b,o}\mathrm{e}^{-\tau_{\nu}} + T(1-\mathrm{e}^{-\tau_{\nu}})
}$$

where\marginpar{-28-}
\begin{itemize}
    \item $T$ is the thermal source temperature (\emph{physical} temperature of the layer)
    \item $T_{b}$ is the brightness temperature of the incident radiation
\end{itemize}
\textcolor{bred}{$T_{b}$ is never greater than $T$!}

\subsection{Einstein coefficients}\footnote{Draine, section 6.1}
\emph{Transition probabilities per unit time}. Three possible processes:
\begin{enumerate}
    \item Spontaneous emission: Einstein $A$ coefficient
    \item Absorption
    \item Stimulated\marginpar{-29-} emission
\end{enumerate}

\subsubsection{Relations between the Einstein coefficients}
Valid under all conditions since they only refer to \emph{atomic} principles
(no collisions, just radiation).
TE: rate of transitions out of state 1 = rate of transitions into state 1
(per unit volume).

In\marginpar{-30-} TE, apply the Boltzmann law.

\subsubsection{Relations between Einstein coefficients and $\kappa_{\nu}$ and $j_{\nu}$}
\paragraph{emission coefficient}
\paragraph{absorption coefficient}

The\marginpar{-31-} second term here for $\kappa_{\nu}$ corresponds to 
\emph{stimulated emission}.


\subsection{Line profile function, $\phi(\nu)$}
\footnote{See RL chapter 10.6 and Draine 6.4}
\subsubsection{Natural line width}

\textcolor{bred}{Key point: A small Einstein coefficient $A$ results
in a \emph{narrow} line.}

The natural line width of most transitions is quite small, and broadening
due to other effects is more important.
\subsubsection{Doppler broadening}
\begin{itemize}
    \item Thermal velocities
    \item Bulk motion (turbulence)
\end{itemize}

and\marginpar{-32-} the profile function is

\subsubsection{Collisional broadening}
$\sim$ Pressure broadening, which is not generally important in the ISM because
the density is so low\ldots mostly occurs in stellar atmospheres.
This still produces a Lorenzian profile, but with:
$${
    \phi(\nu) = \frac{4\Gamma^{2}}{16\pi^{2}(v-v_{o})^{2} + \Gamma^{2}}
}$$
Can be written in terms of a \emph{Voigt function}:

So\marginpar{-33-} the core of the profile is Gaussian due to Doppler broadening,
while the wings are much stronger than expected in a Gaussian profile,
due to the intrinsic line width.


\section{Atomic H in the ISM}
\footnote{Draine Ch 8, 29; Ch 17.1, 17.3}
Wherever HI dominates the ISM, all atoms are found in the \textbf{ground state}
($^{2}$S)(n=1). The next excited level ($^{2}$P) is about 10 eV above the
ground-level. This excitation is very rare and and quickly falls back to
the ground level, so there is no significant population of this level.
For example, consider potential excitation mechanism: collision of H-atom
with cosmic ray particle (lots of energy, probability is once per
10$^{17}$ seconds), ionizes the H-atom. Recombination results in some
atoms winding up in $^{2}$P state. But $A$ coefficient for spontaneous
emission from 2P $\rightarrow$ 2S is 10$^{8}$ s$^{-1}$. Hence, this excitation
process results in ralative population of 2P level of
$10^{-8}/10^{17} = 10^{-25}$!

ISM is too cool for collisions to happen often and cosmic rays are rare.
Possible tracers of HI gas:
\begin{enumerate}
    \item 21 cm HI transition (=hyperfine transition) in emission or absorption.
    \item Lyman absorption lines against hot background stars.
\end{enumerate}
Only the $^{2}$S level is populated. HI is hard to find in the ground state;
fine structure $\rightarrow$ different angular momentum.

\footnote{See Draine Ch 4 (\& 5) on notation of energy levels and
atomic structure.}

\subsection{Excitation and radiative transport for the 21-cm line}
\marginpar{-34-}
Spin of proton and electron:
\begin{itemize}
    \item Parallel (upper energy)
    \item Anti-parallel (lower energy)
\end{itemize}
(Spin is around particle's own axis, not to be confused with angular
momentum). Motions specified by maxwellian velocity distribution, and
collisions dominate the level populations (excite and de-excite).

Energy difference (very small):
\begin{align*}
    h\nu &= 9.4\times10^{-18}\:\mathrm{erg}\\
    \nu &= 1420.4\:\mathrm{MHz}\\
    \lambda &= 21.11\:\mathrm{cm}
\end{align*}

Spontaneous emission probability is \emph{very} small:
$${ A_{kj} = 2.86\times10^{-15}\:\mathrm{sec}^{-1}  }$$
$${ \rightarrow \mathrm{lifetime} = 1.10\times10^{7}\:\mathrm{years} }$$
More frequently, atoms will flip energy states by \emph{collisions}.
These dominate transitions and cause energy levels to flip.\\
{\centering H + H $\rightarrow$ H$_{2}^{*}$ $\rightarrow$ H + H}\\
excited H$_{2}$ molecule (not stable). Chance for collisional excitation
per second: $\gamma n(H)$ [s$^{-1}$].

\begin{center}
    \begin{tabular}{c c}
        $\gamma [\times10^{-11}\;\mathrm{cm}^{3}\mathrm{s}^{-1}]$ & T [K]\\
        0.23 & 10\\
        3.0 & 30\\
        9.5 & 100\\
        16 & 300\\
        25 & 1000
    \end{tabular}
\end{center}
As long as $n(H^{o}) > 10^{-2}$ cm$^{-3}$, $\gamma n_{H} >> A$, so
collisions determine excitation and de-excitation\ldots implies
detailed balancing, so Boltzmann distribution is valid for level
populations.
$${ \frac{n_{1}}{n_{o}} = \frac{g_{1}}{g_{o}}\exp(-h\nu/kT_{k})
    \approx \frac{g_{1}}{g_{o}}(1-\frac{h\nu}{kT_{k}}) \approx \frac{g_{1}}{g_{o}}
}$$
where $T_{k}$ = kinetic gas temperature.





\underline{Derivation}:\marginpar{-35-}

Now\marginpar{-36-} assume that\ldots

In general, in situations where stimulated emission and absorption can
be neglected, we would obtain
$${ \frac{b_{1}}{b_{o}} = \frac{1}{1 + \frac{A_{10}}{n_{H}\gamma_{10}}}
    \approx 1
}$$
Note: Potential other excitation process of upper level for HI:
HI + Ly$\alpha$ photon $\rightarrow$ n=2 level (2P). 2P level might cascade
down to upper 2S hyperfine level.

In principle, one might selectively populate the upper level with this
so-called photon pumping\footnote{D 17.3}.
It turns out that, due to thermilization of the photons this anomalous
population of the levels does not generally occur. (Ly$\alpha$ photons
frequently scatter off A-atom again = LTE situation).

\mar{37}
Since the b's are effectively 1, our \textcolor{bred}{main result}
is that$${
}$$

Are the HI levels always populated mainly by collisions? No, if n(H$^{o}$) drops
low enough, and if the HI is warm it doesn't work as well.

The warm HI in our Galaxy has $n_{H}\sim0.4$ cm $^{-3}$ and $T_{k} \approx$ 8000 K.
But $T_{s}$ = spin temperature, of order $T_{k}$/5, if it weren't for
excitation by Ly$\alpha$ photons.

But it is true that the 3K background radiation field does \emph{not}
significantly disturb the equilibrium set up by collisions, as we
discussed above.

If $n_{H}$ drops to very low values, collisional excitation is ineffective.
But even in that case, $T_{s} \approx T_{k}$ because of the above mentioned
Ly$\alpha$ excitation.

In any case, in general we now obtain for the emission coefficient:
$${
}$$
which is independent of T in most circumstances, and for the absorption
coefficient:
$${
}$$


\mar{38}
Using the relation between Einstein coefficients:

\subsection{Consider simple case of a single layer of gas}

\mar{39}
\paragraph{Consider two cases}
\begin{enumerate}[label={(\roman*)}]
    \item $\tau_{\nu_{o}}(L) >> 1$ Optically thick
    \item $\tau_{\nu_{o}}(L) << 1$ Optically thin
\end{enumerate}
In general, the line profile of HI emission is entirely determined by the velocity
of the atoms, so the assumption of a Gaussian profile is correct.

\mar{40}
\subsection{Observing brightness temperature}

\fcolorbox{black}{white}{
    Aside: \textbf{How do we observe HI?}
}

\mar{41}
\subsection{HI emission and absorption}
Distinguish two cases:
\begin{enumerate}
    \item Absorption by an HI cloud of an extended background source, usually
        a continuum source (e.g.\ AGN)
    \item HI self-absorption (enough of it at same velocity)
\end{enumerate}

\mar{42}
\begin{itemize}
    \item $T_{b,\mathrm{off}} = T_{s}(1-\mathrm{e}^{-\tau_{\nu}})$
        in the direction immediately next to the source
    \item $T_{b,\mathrm{on}} = T_{bo}\mathrm{e}^{-\tau_{\nu}} +
        T_{s}(1-\mathrm{e}^{-\tau_{\nu}})$
        directly toward the source
\end{itemize}

\paragraph{Some practical points}
\begin{enumerate}
    \item Frequency \emph{band} in a number of channels, together spanning a
        range in velocity
    \item The assumption here is that the
    \item A way to
    \item Do we really have only \emph{one} HI cloud along the LOS, and is
        its temperature uniform?
\end{enumerate}

\mar{43}
We observe at each frequency

\textcolor{bred}{$T_{b,\mathrm{off}} \leq T_{s} $}

\paragraph{Discussion of some results}

\mar{44}
Plots

\mar{45}Conclusions:
\begin{enumerate}
    \item The absorption profile $(1-\mathrm{e}^{-\tau_{\nu}})$ is always
    sharper and simpler in structure than the emission line profile.
    \item \mar{46}The variation in spin temperature, $T_{s}$, makes one
    wonder whether there is really only one temperature in the cloud.
    \item The range of $T_{s}$ one finds from absorption line studies is
        50 K $\leq T_{s} \leq$ 1000 K. However, the upper value is only
        a lower limit since we don't see absorption anymore for
        $T_{s} >$ 1000 K.
    \item One can compare 
\end{enumerate}

\mar{47}Some relevant results:
\begin{enumerate}
    \item HI is \emph{not} concentrated in a small number of giant clouds,
        as H$_{2}$ is; estimates of filling factor range from 20-90\%?!
    \item HI occurs roughly 50/50 in two important forms:
        \begin{itemize}[itemsep=0.5mm]
            \item CNM = cold neutral medium $\sim$ 80 K
            \item WNM\footnote{Mebold, 1972} = warm neutral medium $\lesssim$ 8000 K
        \end{itemize}
    \item Clouds are filaments and/or sheets, rather than spheres
    \item At low galactic latitudes, it is not possible to distinguish between
        WNM and CNM because there is too much material along the line of sight.
    \item Problem with different temperatures along the line of sight.
        \mar{48}Then:
\end{enumerate}

Special cases:
\begin{enumerate}
    \item $\tau_{1}>>1$
    \item $\tau_{1}<<1$
    \item $\tau_{1}<<1, \tau_{2}<<1$
\end{enumerate}

\mar{49}Some additional points on HI\footnote{Kulkami \& Heiles}
\begin{itemize}
    \item HI measurements toward Galactic continuum sources provide
        (some) information on the distance to these sources,
        using the velocity of the HI absorption feature to derive
        a kinematical distance. The usual problem of distance
        ambiguity may still be a problem here as well, although the
        total HI column [density?] that is derived from the
        absorption measurements helps a little in distinguishing
        near and far distances.
    \item \textbf{Temperature of WNM}\ldots two methods:
        \begin{enumerate}
            \item HI absorption
            \item UV absorption lines
        \end{enumerate}
        So far: data in agreement with 5000 K $\lesssim T \lesssim$ 8000 K,
        but needs more confirmation.
    \item \textbf{Temperature of CNM}\ldots from absorption spectra.
        \mar{50}The absorption lines are \emph{narrow}.
\end{itemize}

Present research on Milky Way HI concentrates on:
\begin{enumerate}
    \item Origin of high velocity clouds
    \item Properties of smallest clouds/features
    \item Search for HI clouds that may contain dark matter, but no stars
\end{enumerate}

\mar{50a, 50b, 50c}
Images and plots.

\mar{I1}
\section{Atomic structure}
\subsection{Introduction}
\subsection{Hydrogen atom \& hydrogen-like atoms (or ions)}
\mar{I2}\underline{Example}: energy of ground states:
\begin{itemize}
    \item H = 13.6 eV
    \item He$^{+}$ = 54.4 eV
    \item Li$^{2+}$ = 122.5 eV
\end{itemize}

\mar{I3}
\mar{I4}
\mar{I5}
\mar{I6}
\subsection{Electron spin}

\mar{I7}But, since $\vec{J}_{1}$ and $\vec{J}_{2}$ can have different directions,
there are different possible values of $\vec{J}$.

\mar{I8}
\subsection{Spin orbit coupling}
Two possible orientations of electron spin with respect to orbital angular momentum
lead to a doubling of energy levels of H-like atoms (except $s$-levels, where
$\ell=0$).

Lines appear in pairs, close together, called doublets.
Example: for Na D lines, $\lambda$ = 5890\AA{}, 5896\AA{}. (Question: are
atoms with filled shells + one electron also hydrogen-like? Doubling is due
to spin-orbit coupling.)

Physically, the spin-orbit coupling produces an extra energy term for the
electron, proportioned to $\vec{S}\cdot\vec{L}$.

\mar{I10}
\subsection{Atoms with multiple electrons}

\mar{I11}
\mar{I12}
\mar{I13}
\mar{I14}
\mar{I15}
\mar{I16}
\mar{I17}
\mar{I18}
\mar{I19}
\subsection{Transition rules}

\mar{I20}
\subsection{X-ray emission}
\subsection{Zeeman effect}

\mar{I21}

\newpage
\mar{51}
\section{HII regions}
\subsection{Introductory remarks}
Process: hot OB stars emit UV photons that can ionize the surrounding neutral H
(and He) medium. In practice, this rquires stars hotter than $\sim$ 30,000
K, aka.\ spectral type B0 or earlier.

Physics for planetary nebulae is similar, but central stars may be much
hotter (though they are also dimmer because there is ``more stuff'').

UV photons impart energy to gas by ionizing H and He. Excess kinetic energy
of created free electrons is shared with ions and other electrons, heating
the medium. Relevant processes:
\begin{itemize}
    \item photoionization
    \item electron-electron encounters
    \item electron-ion encounters
        \begin{itemize}
            \item Excited ions
            \item Recombination
            \item Bremsstrahlung
        \end{itemize}
\end{itemize}
H atom + photon $h\nu$ $\rightarrow$ p$^{+}$ + e$^{-}$ + E$_{k}$

\textcolor{bred}{Since stars create a continuous stream of photons, a
balance is reached between ionization and recombination.}

Field OB stars: were they born there or travel there somehow? Usually in
groups (OB associations), dense clouds, etc.

Consequences: $\sim 10^{4}$ K. Warm ionized plasma emitting various forms
of radiation: recombination, free-free continuum, bree-bound continuum,
2-photon continuum, collisionally excited forbidden lines from ``metals''.
If there is dust around, get MIR dust continuum. Runaway OB stars (like
teenagers!)

1 km $\equiv$ 1 pc in $10^{6}$ years. (?)

$Q_{o}$ [s$^{-1}$]
= number of ionized photons emitted by OB star per second.
Factor of 100 from O 9.5V to O 3V (luminosity classes).

\mar{52}
\subsection{Str\"{o}mgren Theory}
``It's only a model.''

Classical paper\footnote{Str\"{o}mgren 1939 ApJ 89, 526}

Basic result:
\begin{itemize}
    \item Hot star in a uniform medium will ionize a spherical volume out to a certain radius.
    \item There is a \emph{sharp} boundary from the ionized to the
        surrounding neutral medium.
\end{itemize}
\subsection{Simple derivation}
Pure H nebula, uniform density. Hot star emits $N_{Lyc}$ photons per
second ($Lyc$ = Lyman continuum), all at the \emph{same} frequency, $\nu_{o}$.
The star will ionize the gas, but there is a balance:
\begin{center}
    ionization $\leftrightarrow$ recombination
\end{center}
Stationary ionization eqilibrium characterized by \textit{degree of ionization}
inside $R_{o}$: $x(r) \equiv \cfrac{n_{e}(r)}{n_{H}}$ where $n_{H}$ = \emph{total}
hydrogen density (neutral and ionized).
$0 \le x \le 1$ ($x = 1 \rightarrow$ complete ionization; $x = 0
\rightarrow$ completely neutral). Problem: what is the shape of $x(r)$?

Important quantity: the Str\"{o}mgren radius, $R_{SO}$, defined as$${
    \frac{4}{3}\pi\alpha{n_{H}^{2}}{R_{SO}^{3}} \equiv N_{Lyc}
}$$
$\alpha$ is the recombination coefficient (collisional process
$\sim\rho^{2}$ - more encounters).
The quantity $N_{Lyc}$\footnote{$N_{Lyc} = S_{4}$ in Spitzer's notation.}
gives the total number of recombinations per unit
time (all the magic is here!).\footnote{$S_{4}$ formally should include
contributions from the diffuse radation field.}

\mar{53}\subsubsection{Description of photoionization equilibrium}
\paragraph{Overview:} Consider pure H nebula surrouding a single hot star.
Ionization equilibrium:$${
    n_{H^{o}}\int_{\nu_{1}}^{\infty}{\frac{4\pi{J\nu}}{h\nu}\sigma_{\nu}\mathrm{d}\nu} =
    n_{e}n_{p}\alpha(H,T)
}$$where
\begin{itemize}
    \item $n_{H^{o}}$ = neutral hydrogen density
    \item $J_{\nu}$ = mean intensity of radiation field
    \item $\sigma_{\nu}$ = ionization cross-section for H by photons with
        energy above threshold $h\nu_{01}$
    \item $n_{e}, n_{p}$ = electron and proton densities
    \item $\alpha(H,T)$ recombination coefficient
\end{itemize}

\mar{54}Physical conditions: radiative decay from upper levels to $n=1$ is
quick $\rightarrow$ nearly all neutral hydrogen will be in the ground
level. Photoionization takes place from the ground level, and is balanced
by recombination to excited levels, which then quickly de-excite by
emission of photons.

\mar{55}The summed recombination coefficient ($\alpha$) is the relevant
quantity because recombination counts, irrespective of the level $n$ to
which it happens.

\mar{56}Neutral fraction \emph{inside} HII region\footnote{see Draine
section 15.3}: Consider pure H nebula, and case B recombination (see
later). \emph{Mean energy} of stellar ionization photons is $h\nu$.

\mar{57}Rewrite latter as:$${
}$$

so neutral fraction $x$ is small and hence $x^{2} \approx 1$.

\mar{58}Comments:
\begin{enumerate}
    \item Reality:
    \item Is assumption of a static ionization equilibrium realistic? Nope.
    \item Recombination time scale:
    \item As we will see later, each Lyc photon results in $\sim$ 0.35
        H$\alpha$ photons begin produced in this ionization $\rightarrow$
        recombination equilibrium.
    \item Next: influence of dust on $R_{SO}$
\end{enumerate}

\subsection{The spectrum of an HII region}
\mar{63}
\subsubsection{Continuum Radiation}
\subsubsection{Line Radiation}
\paragraph{Recombination lines}
\paragraph{Collisionally excited lines in heavy elements}

\subsection{Types of HII Regions}
\mar{85}In general:
\begin{itemize}
    \item HII regions are associated with molecular clouds and dark
        nebulae.
\end{itemize}

\subsubsection{``Blister model'' -- cavity inside GMC}
\subsubsection{``Champagne model'' -- half cavity at edge of GMC}
\subsubsection{Compact -- only visible at radio and FIR wavelengths}

\section{Interstellar absorption lines in stellar and quasar spectra}

\end{document}
