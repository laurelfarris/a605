\documentclass[12pt]{article}
%\usepackage[left=2in, top=1in, right=1in, bottom=1in]{geometry}
\usepackage[margin=1in]{geometry}
\setlength{\parindent}{0em}
\setlength{\parskip}{0.75ex}
\usepackage{graphicx}
\usepackage{lipsum}
%\usepackage{enumerate}
\usepackage{enumitem}
\usepackage{amsmath}
%\usepackage{mdwlist}
%\usepackage{color}
\usepackage{xcolor}
\usepackage{ragged2e}

\usepackage{marginnote}
%\xdef\marginnotetextwidth{2in}

\definecolor{bred}{rgb}{0.8, 0.0, 0.0}
\definecolor{cadmiumorange}{rgb}{0.93, 0.53, 0.18}
\definecolor{darkcyan}{rgb}{0.0, 0.55, 0.55}
\definecolor{cadmiumgreen}{rgb}{0.0, 0.42, 0.24}
\definecolor{halayaube}{rgb}{0.4, 0.22, 0.33}

% Sections

\usepackage{titlesec}

%\sectionfont{\fontsize{16}{18}\selectfont}
%\subsectionfont{\fontsize{14}{16}\selectfont}
%\subsubsectionfont{\fontsize{12}{14}\selectfont}

%\usepackage{sectsty}
%\titleformat{<command>}
%   [<shape>]{<format>}{<label>}{<sep>}{<before-code>}{<after-code>}
\titleformat{\section}%
    [hang]
  {\fontsize{16}{18}\selectfont\bfseries\color{cadmiumgreen}} %\filcenter\bfseries\LARGE
  {\Roman{section}.\;}        % label%    {\thesection} %{<label>}
  {0pt}     % sep
  {}        % before code
\titleformat{\subsection}%
  {\fontsize{14}{16}\selectfont\bfseries\color{darkcyan}} %\filcenter\bfseries\LARGE
  {} % label%    {\thesection} %{<label>}
  {0pt}     % sep
  {}        % before code
\titleformat{\subsubsection}%
  {\fontsize{12}{14}\selectfont\bfseries\color{halayaube}} %\filcenter\bfseries\LARGE
  {\arabic{subsection}.\arabic{subsubsection}\;}        % label%    {\thesection} %{<label>}
  {0pt}     % sep
  {}        % before code

%\titlespacing*{⟨command⟩}{⟨left⟩}{⟨before-sep⟩}{⟨after-sep⟩}[⟨right-sep⟩]
%\titlespacing*{\section}{-0.5in}{0ex}{0ex}
%\titlespacing*{\subsection}{0pt}{0.5ex}{-10ex}

% Section references
%\renewcommand{\thesection}{}
%\renewcommand{\thesubsection}{\arabic{subsection}}
%\renewcommand{\thesubsubsection}{\arabic{subsubsection}}

%\setcounter{secnumdepth}{1}


% Lists
\setlist[itemize]{%
    itemsep=-1ex}
\setlist[enumerate]{%
    itemsep=-1ex}
\setlist[description]{%
    itemsep=0ex,
    align=right,}
\renewcommand{\labelitemi}{$\vcenter{\hbox{\tiny$\bullet$}}$}
%\renewcommand{\labelitemi}{{\tiny$\bullet$}}
\definecolor{cadet}{rgb}{0.33, 0.41, 0.47}
\renewcommand{\descriptionlabel}[1]{%
    \ttfamily\textcolor{cadet}{#1}
}
% Verbatim
\usepackage{fancyvrb}  % framebox around verbatim text
\makeatletter
\renewcommand\verbatim@font{\normalfont\small\ttfamily}
\makeatother

\usepackage{listings}
\lstset{% general command to set parameter(s)
    basicstyle=\small, % print whole listing small
    keywordstyle=\color{black}\bfseries\underbar,% underlined bold black keywords % nothing happens
    identifierstyle=,
    commentstyle=\color{white},
    stringstyle=\ttfamily,
    showstringspaces=false % no special string spaces
    }

\usepackage{setspace} % spacing between toc items
\usepackage[toc]{multitoc}
\renewcommand*{\multicolumntoc}{2}
%\setlength{\columnseprule}{0.5pt}

\usepackage{hyperref}
\definecolor{darkpowderblue}{rgb}{0.0, 0.2, 0.6}
\hypersetup{colorlinks=true,
    urlcolor=darkpowderblue,
    linkcolor=black % This may be what links the contents in the first place!
}
\urlstyle{same}

\begin{document}
%\reversemarginpar

\section{Introduction}
\begin{itemize}
  \item Neutral gas, molecular gas, general viewpoint -6-
  \item Dust -8-
  \item Ionized Gas -9-
      \begin{itemize}
          \item photoionization
          \item collisions
          \item cosmic rays
      \end{itemize}
  \item Phases of interstellar gas -12-
  \item Magnetic fields and cosmic rays -13-
  \item Pressure sources -14-
\end{itemize}

\section{Validity of the laws of statistical physics in ISM conditions}
\marginpar{-16-}
Four major laws of statistical physics:
\begin{enumerate}
    \item \textbf{Maxwellian} velocity distribution
    \item \textbf{Boltzmann distribution} of energy levels in atoms and molecules
    \item \textbf{Saha equation} for ionization equilibrium
    \item \textbf{Planck function} for radiation
\end{enumerate}
\marginpar{-19-}These four laws hold under thermodynamic equilibrium (TE)
However, this is not often the case for the ISM\@.
TE does not hold for two reasons:
\begin{enumerate}
    \item It requires \textbf{detailed balancing}
    \item The strong dilution of the radiation field.
\end{enumerate}
\subsection{Detailed Balancing}
Each process is as likely to occur as its inverse. For example, consider the
3727 \AA{} emission from O$^{+}$. This is a forbidden transition (actually a
doublet). The excitation of the electron level occurs through collisions with
electrons, in most conditions in the ISM.

\ldots

\subsection{Statistical equilibrium}
\marginpar{-22-}In general, assume \textbf{statistical equilibrium}:
\marginpar{-23-}$\cfrac{\mathrm{d}n_{i}}{\mathrm{d}t} = 0$

\section{Radiative Transfer -24-}
\section{Radiative transfer equation -26-}
\section{Einstein coefficients -28-}

\section{Line profile function, $\phi(\nu)$}
\reversemarginpar
\marginpar{-31- See RL chapter 10.6 and Draine 6.4}
\subsection{Natural line width}

\textcolor{bred}{Key point: A small Einstein coefficient $A$ results
in a \emph{narrow} line.}

The natural line width of most transitions is quite small, and broadening
due to other effects is more important.
\subsection{Doppler broadening}
\begin{itemize}
    \item Thermal velocities
    \item Bulk motion (turbulence)
\end{itemize}
\subsection{Collisional broadening}
$\sim$ Pressure broadening, which is not generally important in the ISM because
the density is so low\ldots mostly occurs in stellar atmospheres.
This still produces a Lorenzian profile, but with:
$${
    \phi(\nu) = \frac{4\Gamma^{2}}{16\pi^{2}(v-v_{o})^{2} + \Gamma^{2}}
}$$

\section{Atomic H in the ISM}
\marginpar{-33-\\Draine Ch\\8, 29;\\Ch\\17.1, 17.3}
Wherever HI dominates the ISM, all atoms are found in the \textbf{ground state}.

ISM is too cool for collisions to happen often and cosmic rays are rare.
Possible tracers of HI gas:
\begin{enumerate}
    \item 21 cm HI transition (=hyperfine transition) in emission or absorption.
    \item Lyman absorption lines against hot background stars.
\end{enumerate}
Only the $^{2}$S level is populated. HI is hard to find in the ground state;
fine structure $\rightarrow$ different angular momentum.
\marginpar{See Draine Ch 4 (\& 5) on notation of energy levels and
atomic structure.}

\marginpar{-34-}
\subsection{Excitation and radiative transport for the 21-cm line}
Spin of proton and electron:
\begin{itemize}
    \item Parallel (upper energy)
    \item Anti-parallel (lower energy)
\end{itemize}
(Spin is around particle's own axis, not to be confused with angular
momentum). Motions specified by maxwellian velocity distribution, and
collisions dominate the level populations (excite and de-excite).

Energy difference (very small):
\begin{align*}
    h\nu &= 9.4\times10^{-18}\:\mathrm{erg}\\
    \nu &= 1420.4\:\mathrm{MHz}\\
    \lambda &= 21.11\:\mathrm{cm}
\end{align*}

Spontaneous emission probability is \emph{very} small:
$${ A_{kj} = 2.86\times10^{-15}\:\mathrm{sec}^{-1}  }$$
$${ \rightarrow \mathrm{lifetime} = 1.10\times10^{7}\:\mathrm{years} }$$
More frequently, atoms will flip energy states by \emph{collisions}.

\subsection{Simple case of a single layer of gas}

\section{Atomic Structure}
\begin{itemize}
  \item electron spin -I6-
  \item spin-orbit coupling -I8-
  \item atoms with multiple electrons -I10-
  \item transition rules
  \item x-ray emission, Zeeman effect -I20-
\end{itemize}

\section{HII Regions -51-}
\begin{itemize}
    \item Stromgren Theory -52-
    \item HII Region spectra
    \begin{itemize}
        \item Continuum Radiation -63-
        \begin{itemize}
            \item 2-photon
            \item free-free
            \item free-bound
            \item dust
        \end{itemize}
        \item Line Radiation -69-
        \begin{itemize}
            \item Recombination lines -69-
            \begin{itemize}
                \item radio -71-
                \item optical and IR -72-
            \end{itemize}
            \item Collisionally excited lines -79-
        \end{itemize}
    \end{itemize}
\item Types of HII Regions -85-
    \begin{itemize}
        \item ``Blister model'' -- cavity inside GMC
        \item ``Champagne model'' -- half cavity at edge of GMC
        \item Compact -- only visible at radio and FIR wavelengths
    \end{itemize}
\end{itemize}

\section{Spectra in the ISM -89-}
\begin{itemize}
    \item Interstellar absorption lines in stellar and quasar spectra -89-
    \item Theory of formation of (interstellar) absorption lines -90-
        \begin{itemize}
            \item Equivalent width (W) -90-
        \end{itemize}
    \item Growth curves in practice -94-
    \item UV absorption lines from H and H$_2$ -96-
\end{itemize}

\section{Dust -104-}
    \begin{itemize}
  \item Far infrared emission from dust -104-
  \begin{itemize}
    \item General properties of dust -104-
    \item Absorption efficiency: the Q parameter -105-
    \item Calculating dust mass from FIR fluxes -106-
    \begin{itemize}
      \item Spectrum emitted by dust grains $\rightarrow$
      modified blackbody spectrum -108-
    \end{itemize}
    \item Dust temperatures -109-
  \end{itemize}
  \item Interstellar extinction -116-
  \begin{itemize}
    \item The extinction law -117-
  \end{itemize}
  \item Interstellar reddening -118-
    \end{itemize}

\section{Molecular Hydrogen and CO -124-}
      \begin{itemize}
  \item Molecular gas and CO as a tracer -128-
  \item Collisional Excitation and Ionization -142-
  \item Properties of hot ionized gas and spectrum -145-
  \end{itemize}
\section{Heating and Cooling -151-}
    \begin{itemize}
      \item General
      \begin{itemize}
        \item Primary heat source: photoionization -152-
        \item Primary cooling source: inelastic collisions -153-
      \end{itemize}
      \item HII regions (ionized) -154-
      \begin{itemize}
        \item Heating -154-
        \item Cooling -155-
      \end{itemize}
      \item HI gas (neutral) -160-
      \begin{itemize}
        \item Dominant cooling line from CII (IP = 11.26 eV)
        \item HI naturally cool, but observe very warm HI!
        \item General players:
        \begin{itemize}
          \item Cooling function -160-
          \item Heating function -161-
        \end{itemize}
      \end{itemize}
    \end{itemize}

\end{document}
